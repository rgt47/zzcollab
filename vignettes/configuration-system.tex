% Options for packages loaded elsewhere
\PassOptionsToPackage{unicode}{hyperref}
\PassOptionsToPackage{hyphens}{url}
\documentclass[
]{article}
\usepackage{xcolor}
\usepackage[margin=1in]{geometry}
\usepackage{amsmath,amssymb}
\setcounter{secnumdepth}{5}
\usepackage{iftex}
\ifPDFTeX
  \usepackage[T1]{fontenc}
  \usepackage[utf8]{inputenc}
  \usepackage{textcomp} % provide euro and other symbols
\else % if luatex or xetex
  \usepackage{unicode-math} % this also loads fontspec
  \defaultfontfeatures{Scale=MatchLowercase}
  \defaultfontfeatures[\rmfamily]{Ligatures=TeX,Scale=1}
\fi
\usepackage{lmodern}
\ifPDFTeX\else
  % xetex/luatex font selection
\fi
% Use upquote if available, for straight quotes in verbatim environments
\IfFileExists{upquote.sty}{\usepackage{upquote}}{}
\IfFileExists{microtype.sty}{% use microtype if available
  \usepackage[]{microtype}
  \UseMicrotypeSet[protrusion]{basicmath} % disable protrusion for tt fonts
}{}
\makeatletter
\@ifundefined{KOMAClassName}{% if non-KOMA class
  \IfFileExists{parskip.sty}{%
    \usepackage{parskip}
  }{% else
    \setlength{\parindent}{0pt}
    \setlength{\parskip}{6pt plus 2pt minus 1pt}}
}{% if KOMA class
  \KOMAoptions{parskip=half}}
\makeatother
\usepackage{color}
\usepackage{fancyvrb}
\newcommand{\VerbBar}{|}
\newcommand{\VERB}{\Verb[commandchars=\\\{\}]}
\DefineVerbatimEnvironment{Highlighting}{Verbatim}{commandchars=\\\{\}}
% Add ',fontsize=\small' for more characters per line
\usepackage{framed}
\definecolor{shadecolor}{RGB}{248,248,248}
\newenvironment{Shaded}{\begin{snugshade}}{\end{snugshade}}
\newcommand{\AlertTok}[1]{\textcolor[rgb]{0.94,0.16,0.16}{#1}}
\newcommand{\AnnotationTok}[1]{\textcolor[rgb]{0.56,0.35,0.01}{\textbf{\textit{#1}}}}
\newcommand{\AttributeTok}[1]{\textcolor[rgb]{0.13,0.29,0.53}{#1}}
\newcommand{\BaseNTok}[1]{\textcolor[rgb]{0.00,0.00,0.81}{#1}}
\newcommand{\BuiltInTok}[1]{#1}
\newcommand{\CharTok}[1]{\textcolor[rgb]{0.31,0.60,0.02}{#1}}
\newcommand{\CommentTok}[1]{\textcolor[rgb]{0.56,0.35,0.01}{\textit{#1}}}
\newcommand{\CommentVarTok}[1]{\textcolor[rgb]{0.56,0.35,0.01}{\textbf{\textit{#1}}}}
\newcommand{\ConstantTok}[1]{\textcolor[rgb]{0.56,0.35,0.01}{#1}}
\newcommand{\ControlFlowTok}[1]{\textcolor[rgb]{0.13,0.29,0.53}{\textbf{#1}}}
\newcommand{\DataTypeTok}[1]{\textcolor[rgb]{0.13,0.29,0.53}{#1}}
\newcommand{\DecValTok}[1]{\textcolor[rgb]{0.00,0.00,0.81}{#1}}
\newcommand{\DocumentationTok}[1]{\textcolor[rgb]{0.56,0.35,0.01}{\textbf{\textit{#1}}}}
\newcommand{\ErrorTok}[1]{\textcolor[rgb]{0.64,0.00,0.00}{\textbf{#1}}}
\newcommand{\ExtensionTok}[1]{#1}
\newcommand{\FloatTok}[1]{\textcolor[rgb]{0.00,0.00,0.81}{#1}}
\newcommand{\FunctionTok}[1]{\textcolor[rgb]{0.13,0.29,0.53}{\textbf{#1}}}
\newcommand{\ImportTok}[1]{#1}
\newcommand{\InformationTok}[1]{\textcolor[rgb]{0.56,0.35,0.01}{\textbf{\textit{#1}}}}
\newcommand{\KeywordTok}[1]{\textcolor[rgb]{0.13,0.29,0.53}{\textbf{#1}}}
\newcommand{\NormalTok}[1]{#1}
\newcommand{\OperatorTok}[1]{\textcolor[rgb]{0.81,0.36,0.00}{\textbf{#1}}}
\newcommand{\OtherTok}[1]{\textcolor[rgb]{0.56,0.35,0.01}{#1}}
\newcommand{\PreprocessorTok}[1]{\textcolor[rgb]{0.56,0.35,0.01}{\textit{#1}}}
\newcommand{\RegionMarkerTok}[1]{#1}
\newcommand{\SpecialCharTok}[1]{\textcolor[rgb]{0.81,0.36,0.00}{\textbf{#1}}}
\newcommand{\SpecialStringTok}[1]{\textcolor[rgb]{0.31,0.60,0.02}{#1}}
\newcommand{\StringTok}[1]{\textcolor[rgb]{0.31,0.60,0.02}{#1}}
\newcommand{\VariableTok}[1]{\textcolor[rgb]{0.00,0.00,0.00}{#1}}
\newcommand{\VerbatimStringTok}[1]{\textcolor[rgb]{0.31,0.60,0.02}{#1}}
\newcommand{\WarningTok}[1]{\textcolor[rgb]{0.56,0.35,0.01}{\textbf{\textit{#1}}}}
\usepackage{graphicx}
\makeatletter
\newsavebox\pandoc@box
\newcommand*\pandocbounded[1]{% scales image to fit in text height/width
  \sbox\pandoc@box{#1}%
  \Gscale@div\@tempa{\textheight}{\dimexpr\ht\pandoc@box+\dp\pandoc@box\relax}%
  \Gscale@div\@tempb{\linewidth}{\wd\pandoc@box}%
  \ifdim\@tempb\p@<\@tempa\p@\let\@tempa\@tempb\fi% select the smaller of both
  \ifdim\@tempa\p@<\p@\scalebox{\@tempa}{\usebox\pandoc@box}%
  \else\usebox{\pandoc@box}%
  \fi%
}
% Set default figure placement to htbp
\def\fps@figure{htbp}
\makeatother
\setlength{\emergencystretch}{3em} % prevent overfull lines
\providecommand{\tightlist}{%
  \setlength{\itemsep}{0pt}\setlength{\parskip}{0pt}}
\usepackage{fontspec}
\usepackage{xcolor}
\setmainfont{Helvetica Neue}
\setsansfont{Helvetica Neue}
\newfontfamily\emojifont{Noto Color Emoji}
\usepackage{newunicodechar}
\newunicodechar{🐳}{\textbf{[Docker]}}
\newunicodechar{📦}{\textbf{[Package]}}
\newunicodechar{⚙}{\textbf{[Settings]}}
\newunicodechar{✅}{$\checkmark$}
\newunicodechar{🔬}{\textbf{[Science]}}
\newunicodechar{🏔}{\textbf{[Alpine]}}
\newunicodechar{🧪}{\textbf{[Test]}}
\newunicodechar{📊}{\textbf{[Analysis]}}
\newunicodechar{📄}{\textbf{[Document]}}
\newunicodechar{✓}{$\checkmark$}
\usepackage{bookmark}
\IfFileExists{xurl.sty}{\usepackage{xurl}}{} % add URL line breaks if available
\urlstyle{same}
\hypersetup{
  pdftitle={ZZCOLLAB Configuration System Guide},
  pdfauthor={ZZCOLLAB Team},
  hidelinks,
  pdfcreator={LaTeX via pandoc}}

\title{ZZCOLLAB Configuration System Guide}
\usepackage{etoolbox}
\makeatletter
\providecommand{\subtitle}[1]{% add subtitle to \maketitle
  \apptocmd{\@title}{\par {\large #1 \par}}{}{}
}
\makeatother
\subtitle{Advanced Docker and Package Management}
\author{ZZCOLLAB Team}
\date{2025-09-23}

\begin{document}
\maketitle

{
\setcounter{tocdepth}{3}
\tableofcontents
}
\section{Overview}\label{overview}

ZZCOLLAB features a sophisticated multi-layered configuration system
that controls Docker images, R packages, build modes, and team
collaboration settings. This system enables teams to create
reproducible, customized research environments while eliminating
repetitive manual configuration.

\subsection{Key Benefits}\label{key-benefits}

\begin{itemize}
\tightlist
\item
  \textbf{✓ Eliminate repetitive typing}: Set defaults once, use across
  all projects
\item
  \textbf{✓ Team standardization}: Shared configurations ensure
  identical environments
\item
  \textbf{✓ Flexible customization}: 14+ Docker variants, custom package
  lists, build modes
\item
  \textbf{✓ Hierarchical control}: User, project, and system-level
  configurations
\item
  \textbf{✓ Single source of truth}: Centralized variant definitions
  prevent duplication
\end{itemize}

\section{Configuration Architecture}\label{configuration-architecture}

\subsection{Multi-Level Hierarchy}\label{multi-level-hierarchy}

ZZCOLLAB loads configuration from multiple sources in priority order:

\begin{enumerate}
\def\labelenumi{\arabic{enumi}.}
\tightlist
\item
  \textbf{Project config} (\texttt{./zzcollab.yaml}) - Team-specific
  settings for shared projects
\item
  \textbf{User config}
  (\texttt{\textasciitilde{}/.zzcollab/config.yaml}) - Personal defaults
  across all projects
\item
  \textbf{System config} (\texttt{/etc/zzcollab/config.yaml}) -
  Organization-wide defaults
\item
  \textbf{Built-in defaults} - Fallback values ensuring system
  functionality
\end{enumerate}

Higher priority configurations override lower priority ones, allowing
for precise control at each level.

\subsection{Three Configuration
Domains}\label{three-configuration-domains}

\subsubsection{🐳 Docker Variant
Management}\label{docker-variant-management}

Control which specialized Docker environments your team uses: -
\textbf{14+ predefined variants} from minimal (\textasciitilde200MB
Alpine) to comprehensive (\textasciitilde3.5GB full-featured) -
\textbf{Domain-specific environments} for bioinformatics, geospatial
analysis, machine learning - \textbf{Single source of truth}
architecture eliminates duplication - \textbf{Interactive variant
browser} for easy discovery and selection

\subsubsection{📦 Package Management
System}\label{package-management-system}

Control what R packages are installed and when: - \textbf{Build mode
control}: Fast (9 packages) vs Standard (17 packages) vs Comprehensive
(47+ packages) - \textbf{Paradigm-specific packages}: Analysis,
Manuscript, and Package development workflows - \textbf{Custom package
lists}: Define your own combinations for specialized workflows -
\textbf{Dependency management}: Docker packages vs renv packages with
proper separation

\subsubsection{Development \& Collaboration
Settings}\label{development-collaboration-settings}

Control team workflows and automation: - \textbf{GitHub integration}:
Automatic repository creation, visibility settings, Actions enablement -
\textbf{Container defaults}: User accounts, working directories, shared
volumes - \textbf{Development preferences}: Default interfaces, dotfiles
handling, confirmation prompts

\section{Docker Variant System}\label{docker-variant-system}

\subsection{Understanding Variants}\label{understanding-variants}

Docker variants are pre-configured environments with specific base
images, R packages, and system dependencies. Each variant is optimized
for particular research workflows.

\subsubsection{Variant Categories}\label{variant-categories}

\textbf{📦 Standard Research Environments (6 variants)}

\begin{Shaded}
\begin{Highlighting}[]
\FunctionTok{minimal}\KeywordTok{:}\CommentTok{          \# \textasciitilde{}800MB  {-} Essential R packages only}
\FunctionTok{analysis}\KeywordTok{:}\CommentTok{         \# \textasciitilde{}1.2GB  {-} Tidyverse + data analysis tools}
\FunctionTok{modeling}\KeywordTok{:}\CommentTok{         \# \textasciitilde{}1.5GB  {-} Machine learning with tidymodels}
\FunctionTok{publishing}\KeywordTok{:}\CommentTok{       \# \textasciitilde{}3GB    {-} LaTeX, Quarto, bookdown, blogdown}
\FunctionTok{shiny}\KeywordTok{:}\CommentTok{            \# \textasciitilde{}1.8GB  {-} Interactive web applications}
\FunctionTok{shiny\_verse}\KeywordTok{:}\CommentTok{      \# \textasciitilde{}3.5GB  {-} Shiny with tidyverse + publishing}
\end{Highlighting}
\end{Shaded}

\textbf{🔬 Specialized Domains (2 variants)}

\begin{Shaded}
\begin{Highlighting}[]
\FunctionTok{bioinformatics}\KeywordTok{:}\CommentTok{   \# \textasciitilde{}2GB    {-} Bioconductor genomics packages}
\FunctionTok{geospatial}\KeywordTok{:}\CommentTok{       \# \textasciitilde{}2.5GB  {-} sf, terra, leaflet mapping tools}
\end{Highlighting}
\end{Shaded}

\textbf{🏔 Lightweight Alpine Variants (3 variants)}

\begin{Shaded}
\begin{Highlighting}[]
\FunctionTok{alpine\_minimal}\KeywordTok{:}\CommentTok{   \# \textasciitilde{}200MB  {-} Ultra{-}lightweight for CI/CD}
\FunctionTok{alpine\_analysis}\KeywordTok{:}\CommentTok{  \# \textasciitilde{}400MB  {-} Essential analysis in tiny container}
\FunctionTok{hpc\_alpine}\KeywordTok{:}\CommentTok{       \# \textasciitilde{}600MB  {-} High{-}performance parallel processing}
\end{Highlighting}
\end{Shaded}

\textbf{🧪 R-Hub Testing Environments (3 variants)}

\begin{Shaded}
\begin{Highlighting}[]
\FunctionTok{rhub\_ubuntu}\KeywordTok{:}\CommentTok{      \# \textasciitilde{}1GB    {-} CRAN{-}compatible package testing}
\FunctionTok{rhub\_fedora}\KeywordTok{:}\CommentTok{      \# \textasciitilde{}1.2GB  {-} Test against R{-}devel}
\FunctionTok{rhub\_windows}\KeywordTok{:}\CommentTok{     \# \textasciitilde{}1.5GB  {-} Windows compatibility testing}
\end{Highlighting}
\end{Shaded}

\subsection{Single Source of Truth
Architecture}\label{single-source-of-truth-architecture}

\subsubsection{Master Variant Library}\label{master-variant-library}

All variant definitions are stored in
\texttt{templates/variant\_examples.yaml}:

\begin{Shaded}
\begin{Highlighting}[]
\CommentTok{\# Example variant definition}
\FunctionTok{analysis}\KeywordTok{:}
\AttributeTok{  }\FunctionTok{base\_image}\KeywordTok{:}\AttributeTok{ }\StringTok{"rocker/tidyverse:latest"}
\AttributeTok{  }\FunctionTok{description}\KeywordTok{:}\AttributeTok{ }\StringTok{"Data analysis environment with tidyverse and common packages"}
\AttributeTok{  }\FunctionTok{packages}\KeywordTok{:}
\AttributeTok{    }\KeywordTok{{-}}\AttributeTok{ }\StringTok{"renv"}
\AttributeTok{    }\KeywordTok{{-}}\AttributeTok{ }\StringTok{"devtools"}
\AttributeTok{    }\KeywordTok{{-}}\AttributeTok{ }\StringTok{"here"}
\AttributeTok{    }\KeywordTok{{-}}\AttributeTok{ }\StringTok{"janitor"}
\AttributeTok{    }\KeywordTok{{-}}\AttributeTok{ }\StringTok{"scales"}
\AttributeTok{    }\KeywordTok{{-}}\AttributeTok{ }\StringTok{"patchwork"}
\AttributeTok{    }\KeywordTok{{-}}\AttributeTok{ }\StringTok{"gt"}
\AttributeTok{    }\KeywordTok{{-}}\AttributeTok{ }\StringTok{"DT"}
\AttributeTok{  }\FunctionTok{system\_deps}\KeywordTok{:}
\AttributeTok{    }\KeywordTok{{-}}\AttributeTok{ }\StringTok{"libxml2{-}dev"}
\AttributeTok{    }\KeywordTok{{-}}\AttributeTok{ }\StringTok{"libcurl4{-}openssl{-}dev"}
\AttributeTok{    }\KeywordTok{{-}}\AttributeTok{ }\StringTok{"libssl{-}dev"}
\AttributeTok{  }\FunctionTok{category}\KeywordTok{:}\AttributeTok{ }\StringTok{"standard"}
\AttributeTok{  }\FunctionTok{size}\KeywordTok{:}\AttributeTok{ }\StringTok{"\textasciitilde{}1.2GB"}
\end{Highlighting}
\end{Shaded}

\subsubsection{Team Configuration
Selection}\label{team-configuration-selection}

Teams select which variants to enable in their \texttt{config.yaml}:

\begin{Shaded}
\begin{Highlighting}[]
\FunctionTok{variants}\KeywordTok{:}
\AttributeTok{  }\FunctionTok{analysis}\KeywordTok{:}
\AttributeTok{    }\FunctionTok{enabled}\KeywordTok{:}\AttributeTok{ }\CharTok{true}\CommentTok{    \# Enable this variant for the team}
\CommentTok{    \# Full definition automatically pulled from variant\_examples.yaml}

\AttributeTok{  }\FunctionTok{modeling}\KeywordTok{:}
\AttributeTok{    }\FunctionTok{enabled}\KeywordTok{:}\AttributeTok{ }\CharTok{false}\CommentTok{   \# Disable this variant (won\textquotesingle{}t be built)}

\AttributeTok{  }\FunctionTok{custom\_ml}\KeywordTok{:}
\AttributeTok{    }\FunctionTok{enabled}\KeywordTok{:}\AttributeTok{ }\CharTok{true}
\CommentTok{    \# Can override or extend definitions from the library}
\AttributeTok{    }\FunctionTok{packages}\KeywordTok{:}\AttributeTok{ }\KeywordTok{[}\StringTok{"additional\_package"}\KeywordTok{]}
\end{Highlighting}
\end{Shaded}

\subsection{Interactive Variant
Management}\label{interactive-variant-management}

\subsubsection{Using the Variant
Browser}\label{using-the-variant-browser}

The \texttt{./add\_variant.sh} script provides an interactive interface
for variant selection:

\begin{Shaded}
\begin{Highlighting}[]
\BuiltInTok{cd}\NormalTok{ your{-}project}
\ExtensionTok{./add\_variant.sh}
\end{Highlighting}
\end{Shaded}

This displays a categorized menu:

\begin{verbatim}
STANDARD RESEARCH ENVIRONMENTS
 1) minimal          ~800MB  - Essential R packages only
 2) analysis         ~1.2GB  - Tidyverse + data analysis
 3) modeling         ~1.5GB  - Machine learning with tidymodels

SPECIALIZED DOMAINS
 7) bioinformatics   ~2GB    - Bioconductor genomics packages
 8) geospatial       ~2.5GB  - sf, terra, leaflet mapping

LIGHTWEIGHT ALPINE VARIANTS
 9) alpine_minimal   ~200MB  - Ultra-lightweight CI/CD
10) alpine_analysis  ~400MB  - Essential analysis in tiny container

Select variants to add (e.g., 1,3,9):
\end{verbatim}

Selected variants are automatically added to your \texttt{config.yaml}
with \texttt{enabled:\ true}.

\subsubsection{Manual Variant
Management}\label{manual-variant-management}

You can also edit variants directly in \texttt{config.yaml}:

\begin{Shaded}
\begin{Highlighting}[]
\FunctionTok{variants}\KeywordTok{:}
\CommentTok{  \# Standard environments}
\AttributeTok{  }\FunctionTok{minimal}\KeywordTok{:}
\AttributeTok{    }\FunctionTok{enabled}\KeywordTok{:}\AttributeTok{ }\CharTok{true}\CommentTok{     \# Essential development}
\AttributeTok{  }\FunctionTok{analysis}\KeywordTok{:}
\AttributeTok{    }\FunctionTok{enabled}\KeywordTok{:}\AttributeTok{ }\CharTok{true}\CommentTok{     \# Primary data analysis}
\AttributeTok{  }\FunctionTok{modeling}\KeywordTok{:}
\AttributeTok{    }\FunctionTok{enabled}\KeywordTok{:}\AttributeTok{ }\CharTok{false}\CommentTok{    \# Machine learning (disabled)}

\CommentTok{  \# Specialized environments}
\AttributeTok{  }\FunctionTok{bioinformatics}\KeywordTok{:}
\AttributeTok{    }\FunctionTok{enabled}\KeywordTok{:}\AttributeTok{ }\CharTok{true}\CommentTok{     \# Genomics analysis}
\AttributeTok{  }\FunctionTok{alpine\_minimal}\KeywordTok{:}
\AttributeTok{    }\FunctionTok{enabled}\KeywordTok{:}\AttributeTok{ }\CharTok{true}\CommentTok{     \# Lightweight CI/CD}
\end{Highlighting}
\end{Shaded}

\subsection{Custom Variant Creation}\label{custom-variant-creation}

\subsubsection{Extending Existing
Variants}\label{extending-existing-variants}

Add packages to existing variants:

\begin{Shaded}
\begin{Highlighting}[]
\FunctionTok{variants}\KeywordTok{:}
\AttributeTok{  }\FunctionTok{analysis}\KeywordTok{:}
\AttributeTok{    }\FunctionTok{enabled}\KeywordTok{:}\AttributeTok{ }\CharTok{true}
\CommentTok{    \# Extend the base analysis variant}
\AttributeTok{    }\FunctionTok{additional\_packages}\KeywordTok{:}
\AttributeTok{      }\KeywordTok{{-}}\AttributeTok{ }\StringTok{"pins"}
\AttributeTok{      }\KeywordTok{{-}}\AttributeTok{ }\StringTok{"vetiver"}
\AttributeTok{      }\KeywordTok{{-}}\AttributeTok{ }\StringTok{"plumber"}
\end{Highlighting}
\end{Shaded}

\subsubsection{Creating New Variants}\label{creating-new-variants}

Define completely custom variants:

\begin{Shaded}
\begin{Highlighting}[]
\FunctionTok{variants}\KeywordTok{:}
\AttributeTok{  }\FunctionTok{custom\_forecast}\KeywordTok{:}
\AttributeTok{    }\FunctionTok{enabled}\KeywordTok{:}\AttributeTok{ }\CharTok{true}
\AttributeTok{    }\FunctionTok{base\_image}\KeywordTok{:}\AttributeTok{ }\StringTok{"rocker/r{-}ver:latest"}
\AttributeTok{    }\FunctionTok{description}\KeywordTok{:}\AttributeTok{ }\StringTok{"Time series forecasting environment"}
\AttributeTok{    }\FunctionTok{packages}\KeywordTok{:}
\AttributeTok{      }\KeywordTok{{-}}\AttributeTok{ }\StringTok{"renv"}
\AttributeTok{      }\KeywordTok{{-}}\AttributeTok{ }\StringTok{"devtools"}
\AttributeTok{      }\KeywordTok{{-}}\AttributeTok{ }\StringTok{"tidyverse"}
\AttributeTok{      }\KeywordTok{{-}}\AttributeTok{ }\StringTok{"forecast"}
\AttributeTok{      }\KeywordTok{{-}}\AttributeTok{ }\StringTok{"fable"}
\AttributeTok{      }\KeywordTok{{-}}\AttributeTok{ }\StringTok{"tsibble"}
\AttributeTok{      }\KeywordTok{{-}}\AttributeTok{ }\StringTok{"feasts"}
\AttributeTok{    }\FunctionTok{system\_deps}\KeywordTok{:}
\AttributeTok{      }\KeywordTok{{-}}\AttributeTok{ }\StringTok{"libxml2{-}dev"}
\AttributeTok{      }\KeywordTok{{-}}\AttributeTok{ }\StringTok{"libssl{-}dev"}
\AttributeTok{    }\FunctionTok{category}\KeywordTok{:}\AttributeTok{ }\StringTok{"custom"}
\AttributeTok{    }\FunctionTok{size}\KeywordTok{:}\AttributeTok{ }\StringTok{"\textasciitilde{}1.3GB"}
\end{Highlighting}
\end{Shaded}

\section{Package Management System}\label{package-management-system-1}

\subsection{Build Mode Package
Control}\label{build-mode-package-control}

ZZCOLLAB provides three build modes that control the number and type of
packages installed:

\subsubsection{Fast Mode (-F): 9 Essential
Packages}\label{fast-mode--f-9-essential-packages}

\begin{Shaded}
\begin{Highlighting}[]
\CommentTok{\# Core development packages only}
\NormalTok{packages }\OtherTok{\textless{}{-}} \FunctionTok{c}\NormalTok{(}
  \StringTok{"renv"}\NormalTok{, }\StringTok{"here"}\NormalTok{, }\StringTok{"usethis"}\NormalTok{, }\StringTok{"devtools"}\NormalTok{,}
  \StringTok{"testthat"}\NormalTok{, }\StringTok{"knitr"}\NormalTok{, }\StringTok{"rmarkdown"}\NormalTok{, }\StringTok{"targets"}
\NormalTok{)}
\end{Highlighting}
\end{Shaded}

\textbf{Use cases}: - Quick prototyping and testing - CI/CD pipelines
requiring fast builds - Resource-constrained environments - Initial
project setup

\subsubsection{Standard Mode (-S): 17 Balanced Packages
(Default)}\label{standard-mode--s-17-balanced-packages-default}

\begin{Shaded}
\begin{Highlighting}[]
\CommentTok{\# Essential + common analysis packages}
\NormalTok{packages }\OtherTok{\textless{}{-}} \FunctionTok{c}\NormalTok{(}
  \CommentTok{\# Fast mode packages}
  \StringTok{"renv"}\NormalTok{, }\StringTok{"here"}\NormalTok{, }\StringTok{"usethis"}\NormalTok{, }\StringTok{"devtools"}\NormalTok{, }\StringTok{"testthat"}\NormalTok{,}
  \StringTok{"knitr"}\NormalTok{, }\StringTok{"rmarkdown"}\NormalTok{, }\StringTok{"targets"}\NormalTok{,}
  \CommentTok{\# Additional analysis packages}
  \StringTok{"dplyr"}\NormalTok{, }\StringTok{"ggplot2"}\NormalTok{, }\StringTok{"tidyr"}\NormalTok{, }\StringTok{"palmerpenguins"}\NormalTok{,}
  \StringTok{"broom"}\NormalTok{, }\StringTok{"janitor"}\NormalTok{, }\StringTok{"DT"}\NormalTok{, }\StringTok{"conflicted"}
\NormalTok{)}
\end{Highlighting}
\end{Shaded}

\textbf{Use cases}: - Most research projects and data analysis -
Balanced functionality vs build time - Team development environments -
Standard exploratory data analysis

\subsubsection{Comprehensive Mode (-C): 47+ Full
Packages}\label{comprehensive-mode--c-47-full-packages}

\begin{Shaded}
\begin{Highlighting}[]
\CommentTok{\# Complete development and analysis ecosystem}
\NormalTok{packages }\OtherTok{\textless{}{-}} \FunctionTok{c}\NormalTok{(}
  \CommentTok{\# Standard mode packages + advanced tools}
  \StringTok{"tidymodels"}\NormalTok{, }\StringTok{"shiny"}\NormalTok{, }\StringTok{"plotly"}\NormalTok{, }\StringTok{"quarto"}\NormalTok{,}
  \StringTok{"flexdashboard"}\NormalTok{, }\StringTok{"survival"}\NormalTok{, }\StringTok{"lme4"}\NormalTok{, }\StringTok{"car"}\NormalTok{,}
  \StringTok{"naniar"}\NormalTok{, }\StringTok{"skimr"}\NormalTok{, }\StringTok{"visdat"}\NormalTok{, }\StringTok{"ggthemes"}\NormalTok{,}
  \StringTok{"kableExtra"}\NormalTok{, }\StringTok{"bookdown"}\NormalTok{, }\StringTok{"jsonlite"}\NormalTok{, }\StringTok{"datapasta"}\NormalTok{,}
  \CommentTok{\# Database and web packages}
  \StringTok{"DBI"}\NormalTok{, }\StringTok{"RMariaDB"}\NormalTok{, }\StringTok{"RPostgreSQL"}\NormalTok{, }\StringTok{"httr"}\NormalTok{, }\StringTok{"rvest"}
\NormalTok{)}
\end{Highlighting}
\end{Shaded}

\textbf{Use cases}: - Comprehensive research environments -
Multi-paradigm projects (analysis + modeling + publishing) - Teaching
environments with diverse package needs - Production analysis
environments

\subsection{Paradigm-Specific Package
Selection}\label{paradigm-specific-package-selection}

\subsubsection{Analysis Paradigm
Packages}\label{analysis-paradigm-packages}

Optimized for data exploration and statistical analysis:

\begin{Shaded}
\begin{Highlighting}[]
\NormalTok{analysis\_packages }\OtherTok{\textless{}{-}} \FunctionTok{c}\NormalTok{(}
  \CommentTok{\# Data manipulation and visualization}
  \StringTok{"tidyverse"}\NormalTok{, }\StringTok{"here"}\NormalTok{, }\StringTok{"janitor"}\NormalTok{, }\StringTok{"skimr"}\NormalTok{, }\StringTok{"naniar"}\NormalTok{,}
  \CommentTok{\# Analysis and modeling}
  \StringTok{"broom"}\NormalTok{, }\StringTok{"infer"}\NormalTok{, }\StringTok{"moderndive"}\NormalTok{,}
  \CommentTok{\# Reporting and communication}
  \StringTok{"targets"}\NormalTok{, }\StringTok{"rmarkdown"}\NormalTok{, }\StringTok{"flexdashboard"}\NormalTok{, }\StringTok{"DT"}\NormalTok{, }\StringTok{"plotly"}
\NormalTok{)}
\end{Highlighting}
\end{Shaded}

\subsubsection{Manuscript Paradigm
Packages}\label{manuscript-paradigm-packages}

Focused on academic writing and reproducible research:

\begin{Shaded}
\begin{Highlighting}[]
\NormalTok{manuscript\_packages }\OtherTok{\textless{}{-}} \FunctionTok{c}\NormalTok{(}
  \CommentTok{\# Core development and testing}
  \StringTok{"devtools"}\NormalTok{, }\StringTok{"testthat"}\NormalTok{, }\StringTok{"roxygen2"}\NormalTok{,}
  \CommentTok{\# Academic writing tools}
  \StringTok{"rmarkdown"}\NormalTok{, }\StringTok{"bookdown"}\NormalTok{, }\StringTok{"papaja"}\NormalTok{, }\StringTok{"RefManageR"}\NormalTok{,}
  \CommentTok{\# Analysis support}
  \StringTok{"here"}\NormalTok{, }\StringTok{"conflicted"}\NormalTok{, }\StringTok{"broom"}
\NormalTok{)}
\end{Highlighting}
\end{Shaded}

\subsubsection{Package Paradigm
Packages}\label{package-paradigm-packages}

Designed for R package development:

\begin{Shaded}
\begin{Highlighting}[]
\NormalTok{package\_packages }\OtherTok{\textless{}{-}} \FunctionTok{c}\NormalTok{(}
  \CommentTok{\# Package development core}
  \StringTok{"devtools"}\NormalTok{, }\StringTok{"usethis"}\NormalTok{, }\StringTok{"roxygen2"}\NormalTok{, }\StringTok{"testthat"}\NormalTok{,}
  \CommentTok{\# Documentation and quality}
  \StringTok{"pkgdown"}\NormalTok{, }\StringTok{"covr"}\NormalTok{, }\StringTok{"lintr"}\NormalTok{, }\StringTok{"styler"}\NormalTok{,}
  \CommentTok{\# Distribution and checking}
  \StringTok{"rcmdcheck"}\NormalTok{, }\StringTok{"rhub"}\NormalTok{, }\StringTok{"spelling"}
\NormalTok{)}
\end{Highlighting}
\end{Shaded}

\subsection{Custom Package Lists}\label{custom-package-lists}

\subsubsection{User-Level Custom
Packages}\label{user-level-custom-packages}

Define personal package preferences in
\texttt{\textasciitilde{}/.zzcollab/config.yaml}:

\begin{Shaded}
\begin{Highlighting}[]
\FunctionTok{build\_modes}\KeywordTok{:}
\AttributeTok{  }\FunctionTok{my\_datascience}\KeywordTok{:}
\AttributeTok{    }\FunctionTok{description}\KeywordTok{:}\AttributeTok{ }\StringTok{"Personal data science workflow"}
\AttributeTok{    }\FunctionTok{docker\_packages}\KeywordTok{:}
\AttributeTok{      }\KeywordTok{{-}}\AttributeTok{ }\StringTok{"renv"}
\AttributeTok{      }\KeywordTok{{-}}\AttributeTok{ }\StringTok{"tidyverse"}
\AttributeTok{      }\KeywordTok{{-}}\AttributeTok{ }\StringTok{"targets"}
\AttributeTok{      }\KeywordTok{{-}}\AttributeTok{ }\StringTok{"pins"}
\AttributeTok{    }\FunctionTok{renv\_packages}\KeywordTok{:}
\AttributeTok{      }\KeywordTok{{-}}\AttributeTok{ }\StringTok{"renv"}
\AttributeTok{      }\KeywordTok{{-}}\AttributeTok{ }\StringTok{"tidyverse"}
\AttributeTok{      }\KeywordTok{{-}}\AttributeTok{ }\StringTok{"targets"}
\AttributeTok{      }\KeywordTok{{-}}\AttributeTok{ }\StringTok{"pins"}
\AttributeTok{      }\KeywordTok{{-}}\AttributeTok{ }\StringTok{"vetiver"}
\AttributeTok{      }\KeywordTok{{-}}\AttributeTok{ }\StringTok{"plumber"}
\AttributeTok{      }\KeywordTok{{-}}\AttributeTok{ }\StringTok{"shiny"}
\AttributeTok{      }\KeywordTok{{-}}\AttributeTok{ }\StringTok{"shinydashboard"}

\AttributeTok{  }\FunctionTok{my\_stats}\KeywordTok{:}
\AttributeTok{    }\FunctionTok{description}\KeywordTok{:}\AttributeTok{ }\StringTok{"Statistical analysis focus"}
\AttributeTok{    }\FunctionTok{docker\_packages}\KeywordTok{:}
\AttributeTok{      }\KeywordTok{{-}}\AttributeTok{ }\StringTok{"renv"}
\AttributeTok{      }\KeywordTok{{-}}\AttributeTok{ }\StringTok{"tidyverse"}
\AttributeTok{      }\KeywordTok{{-}}\AttributeTok{ }\StringTok{"broom"}
\AttributeTok{    }\FunctionTok{renv\_packages}\KeywordTok{:}
\AttributeTok{      }\KeywordTok{{-}}\AttributeTok{ }\StringTok{"renv"}
\AttributeTok{      }\KeywordTok{{-}}\AttributeTok{ }\StringTok{"tidyverse"}
\AttributeTok{      }\KeywordTok{{-}}\AttributeTok{ }\StringTok{"broom"}
\AttributeTok{      }\KeywordTok{{-}}\AttributeTok{ }\StringTok{"lme4"}
\AttributeTok{      }\KeywordTok{{-}}\AttributeTok{ }\StringTok{"survival"}
\AttributeTok{      }\KeywordTok{{-}}\AttributeTok{ }\StringTok{"car"}
\AttributeTok{      }\KeywordTok{{-}}\AttributeTok{ }\StringTok{"emmeans"}
\end{Highlighting}
\end{Shaded}

\subsubsection{Project-Level Custom
Packages}\label{project-level-custom-packages}

Override defaults for specific projects in \texttt{./zzcollab.yaml}:

\begin{Shaded}
\begin{Highlighting}[]
\FunctionTok{build\_modes}\KeywordTok{:}
\AttributeTok{  }\FunctionTok{project\_specific}\KeywordTok{:}
\AttributeTok{    }\FunctionTok{description}\KeywordTok{:}\AttributeTok{ }\StringTok{"Customer churn analysis packages"}
\AttributeTok{    }\FunctionTok{docker\_packages}\KeywordTok{:}
\AttributeTok{      }\KeywordTok{{-}}\AttributeTok{ }\StringTok{"renv"}
\AttributeTok{      }\KeywordTok{{-}}\AttributeTok{ }\StringTok{"tidyverse"}
\AttributeTok{      }\KeywordTok{{-}}\AttributeTok{ }\StringTok{"tidymodels"}
\AttributeTok{    }\FunctionTok{renv\_packages}\KeywordTok{:}
\AttributeTok{      }\KeywordTok{{-}}\AttributeTok{ }\StringTok{"renv"}
\AttributeTok{      }\KeywordTok{{-}}\AttributeTok{ }\StringTok{"tidyverse"}
\AttributeTok{      }\KeywordTok{{-}}\AttributeTok{ }\StringTok{"tidymodels"}
\AttributeTok{      }\KeywordTok{{-}}\AttributeTok{ }\StringTok{"probably"}
\AttributeTok{      }\KeywordTok{{-}}\AttributeTok{ }\StringTok{"workflowsets"}
\AttributeTok{      }\KeywordTok{{-}}\AttributeTok{ }\StringTok{"tune"}
\AttributeTok{      }\KeywordTok{{-}}\AttributeTok{ }\StringTok{"recipes"}
\AttributeTok{      }\KeywordTok{{-}}\AttributeTok{ }\StringTok{"parsnip"}
\AttributeTok{      }\KeywordTok{{-}}\AttributeTok{ }\StringTok{"yardstick"}
\end{Highlighting}
\end{Shaded}

\subsubsection{Package Installation
Separation}\label{package-installation-separation}

ZZCOLLAB distinguishes between two types of package installations:

\textbf{Docker Packages} (\texttt{docker\_packages}): - Installed during
Docker image build - Available immediately when container starts -
Shared across all users of the image - Should include essential packages
needed by everyone

\textbf{renv Packages} (\texttt{renv\_packages}): - Managed by renv for
reproducibility - Project-specific and version-locked - Include
analysis-specific packages - Allow for precise dependency management

\begin{Shaded}
\begin{Highlighting}[]
\FunctionTok{build\_modes}\KeywordTok{:}
\AttributeTok{  }\FunctionTok{optimal\_separation}\KeywordTok{:}
\AttributeTok{    }\FunctionTok{description}\KeywordTok{:}\AttributeTok{ }\StringTok{"Proper package separation example"}
\AttributeTok{    }\FunctionTok{docker\_packages}\KeywordTok{:}
\CommentTok{      \# Essential system packages everyone needs}
\AttributeTok{      }\KeywordTok{{-}}\AttributeTok{ }\StringTok{"renv"}\CommentTok{          \# Package management}
\AttributeTok{      }\KeywordTok{{-}}\AttributeTok{ }\StringTok{"devtools"}\CommentTok{      \# Development tools}
\AttributeTok{      }\KeywordTok{{-}}\AttributeTok{ }\StringTok{"here"}\CommentTok{          \# Path management}
\AttributeTok{      }\KeywordTok{{-}}\AttributeTok{ }\StringTok{"conflicted"}\CommentTok{    \# Namespace conflicts}
\AttributeTok{    }\FunctionTok{renv\_packages}\KeywordTok{:}
\CommentTok{      \# Project{-}specific analysis packages}
\AttributeTok{      }\KeywordTok{{-}}\AttributeTok{ }\StringTok{"renv"}
\AttributeTok{      }\KeywordTok{{-}}\AttributeTok{ }\StringTok{"tidyverse"}\CommentTok{     \# Data analysis}
\AttributeTok{      }\KeywordTok{{-}}\AttributeTok{ }\StringTok{"targets"}\CommentTok{       \# Workflow management}
\AttributeTok{      }\KeywordTok{{-}}\AttributeTok{ }\StringTok{"palmerpenguins"}\CommentTok{ \# Example data}
\AttributeTok{      }\KeywordTok{{-}}\AttributeTok{ }\StringTok{"specific\_analysis\_package"}
\end{Highlighting}
\end{Shaded}

\section{Team Collaboration
Configuration}\label{team-collaboration-configuration}

\subsection{GitHub Integration}\label{github-integration}

ZZCOLLAB provides extensive GitHub integration through configuration:

\subsubsection{Repository Management}\label{repository-management}

\begin{Shaded}
\begin{Highlighting}[]
\FunctionTok{collaboration}\KeywordTok{:}
\AttributeTok{  }\FunctionTok{github}\KeywordTok{:}
\AttributeTok{    }\FunctionTok{auto\_create\_repo}\KeywordTok{:}\AttributeTok{ }\CharTok{true}\CommentTok{          \# Automatically create GitHub repositories}
\AttributeTok{    }\FunctionTok{default\_visibility}\KeywordTok{:}\AttributeTok{ }\StringTok{"private"}\CommentTok{   \# Repository visibility (private/public)}
\AttributeTok{    }\FunctionTok{enable\_actions}\KeywordTok{:}\AttributeTok{ }\CharTok{true}\CommentTok{            \# Enable GitHub Actions CI/CD}
\AttributeTok{    }\FunctionTok{branch\_protection}\KeywordTok{:}\AttributeTok{ }\CharTok{true}\CommentTok{         \# Enable branch protection rules}

\CommentTok{    \# Default repository settings}
\AttributeTok{    }\FunctionTok{repository}\KeywordTok{:}
\AttributeTok{      }\FunctionTok{allow\_merge\_commits}\KeywordTok{:}\AttributeTok{ }\CharTok{true}
\AttributeTok{      }\FunctionTok{allow\_squash\_merging}\KeywordTok{:}\AttributeTok{ }\CharTok{true}
\AttributeTok{      }\FunctionTok{allow\_rebase\_merging}\KeywordTok{:}\AttributeTok{ }\CharTok{false}
\AttributeTok{      }\FunctionTok{delete\_head\_branches}\KeywordTok{:}\AttributeTok{ }\CharTok{true}
\end{Highlighting}
\end{Shaded}

\subsubsection{Issue and PR Templates}\label{issue-and-pr-templates}

\begin{Shaded}
\begin{Highlighting}[]
\FunctionTok{collaboration}\KeywordTok{:}
\AttributeTok{  }\FunctionTok{github}\KeywordTok{:}
\AttributeTok{    }\FunctionTok{templates}\KeywordTok{:}
\CommentTok{      \# Enable issue templates}
\AttributeTok{      }\FunctionTok{issue\_templates}\KeywordTok{:}\AttributeTok{ }\CharTok{true}

\CommentTok{      \# Enable pull request template}
\AttributeTok{      }\FunctionTok{pr\_template}\KeywordTok{:}\AttributeTok{ }\CharTok{true}

\CommentTok{      \# Custom template directory}
\AttributeTok{      }\FunctionTok{template\_dir}\KeywordTok{:}\AttributeTok{ }\StringTok{".github/ISSUE\_TEMPLATE"}
\end{Highlighting}
\end{Shaded}

\subsection{Development Environment
Defaults}\label{development-environment-defaults}

\subsubsection{Container Configuration}\label{container-configuration}

\begin{Shaded}
\begin{Highlighting}[]
\FunctionTok{development}\KeywordTok{:}
\CommentTok{  \# Default variant for team members joining}
\AttributeTok{  }\FunctionTok{default\_interface}\KeywordTok{:}\AttributeTok{ }\StringTok{"analysis"}

\CommentTok{  \# Container user and environment settings}
\AttributeTok{  }\FunctionTok{container}\KeywordTok{:}
\AttributeTok{    }\FunctionTok{default\_user}\KeywordTok{:}\AttributeTok{ }\StringTok{"analyst"}
\AttributeTok{    }\FunctionTok{working\_dir}\KeywordTok{:}\AttributeTok{ }\StringTok{"/home/analyst/project"}
\AttributeTok{    }\FunctionTok{shared\_volumes}\KeywordTok{:}
\AttributeTok{      }\KeywordTok{{-}}\AttributeTok{ }\StringTok{"$\{PWD\}:/home/analyst/project"}
\AttributeTok{      }\KeywordTok{{-}}\AttributeTok{ }\StringTok{"$\{HOME\}/.ssh:/home/analyst/.ssh:ro"}

\CommentTok{  \# Dotfiles handling}
\AttributeTok{  }\FunctionTok{dotfiles}\KeywordTok{:}
\AttributeTok{    }\FunctionTok{auto\_filter\_macos}\KeywordTok{:}\AttributeTok{ }\CharTok{true}\CommentTok{      \# Filter macOS{-}specific commands for Docker}
\AttributeTok{    }\FunctionTok{preserve\_permissions}\KeywordTok{:}\AttributeTok{ }\CharTok{true}\CommentTok{   \# Maintain file permissions}
\AttributeTok{    }\FunctionTok{include\_hidden}\KeywordTok{:}\AttributeTok{ }\CharTok{true}\CommentTok{         \# Include hidden dotfiles}
\end{Highlighting}
\end{Shaded}

\subsubsection{Build and Platform
Settings}\label{build-and-platform-settings}

\begin{Shaded}
\begin{Highlighting}[]
\FunctionTok{build}\KeywordTok{:}
\CommentTok{  \# Docker build configuration}
\AttributeTok{  }\FunctionTok{docker}\KeywordTok{:}
\AttributeTok{    }\FunctionTok{platform}\KeywordTok{:}\AttributeTok{ }\StringTok{"auto"}\CommentTok{            \# auto, linux/amd64, linux/arm64}
\AttributeTok{    }\FunctionTok{no\_cache}\KeywordTok{:}\AttributeTok{ }\CharTok{false}\CommentTok{             \# Enable Docker build caching}
\AttributeTok{    }\FunctionTok{parallel\_builds}\KeywordTok{:}\AttributeTok{ }\CharTok{true}\CommentTok{       \# Build variants in parallel}
\AttributeTok{    }\FunctionTok{memory\_limit}\KeywordTok{:}\AttributeTok{ }\StringTok{"4g"}\CommentTok{          \# Memory limit for builds}

\CommentTok{  \# Package installation settings}
\AttributeTok{  }\FunctionTok{packages}\KeywordTok{:}
\AttributeTok{    }\FunctionTok{repos}\KeywordTok{:}\AttributeTok{ }\StringTok{"https://cran.rstudio.com/"}
\AttributeTok{    }\FunctionTok{install\_suggests}\KeywordTok{:}\AttributeTok{ }\CharTok{false}
\AttributeTok{    }\FunctionTok{dependencies}\KeywordTok{:}\AttributeTok{ }\KeywordTok{[}\StringTok{"Depends"}\KeywordTok{,}\AttributeTok{ }\StringTok{"Imports"}\KeywordTok{,}\AttributeTok{ }\StringTok{"LinkingTo"}\KeywordTok{]}
\AttributeTok{    }\FunctionTok{parallel\_installs}\KeywordTok{:}\AttributeTok{ }\CharTok{true}
\end{Highlighting}
\end{Shaded}

\subsection{Documentation Settings}\label{documentation-settings}

\subsubsection{Automatic Documentation
Generation}\label{automatic-documentation-generation}

\begin{Shaded}
\begin{Highlighting}[]
\FunctionTok{collaboration}\KeywordTok{:}
\AttributeTok{  }\FunctionTok{documentation}\KeywordTok{:}
\AttributeTok{    }\FunctionTok{auto\_generate\_readme}\KeywordTok{:}\AttributeTok{ }\CharTok{true}\CommentTok{       \# Generate project README}
\AttributeTok{    }\FunctionTok{include\_variant\_docs}\KeywordTok{:}\AttributeTok{ }\CharTok{true}\CommentTok{       \# Document available variants}
\AttributeTok{    }\FunctionTok{update\_user\_guide}\KeywordTok{:}\AttributeTok{ }\CharTok{true}\CommentTok{          \# Include ZZCOLLAB\_USER\_GUIDE.md}

\CommentTok{    \# Documentation templates}
\AttributeTok{    }\FunctionTok{readme\_template}\KeywordTok{:}\AttributeTok{ }\StringTok{"README\_TEMPLATE.md"}
\AttributeTok{    }\FunctionTok{contributing\_guide}\KeywordTok{:}\AttributeTok{ }\CharTok{true}
\AttributeTok{    }\FunctionTok{code\_of\_conduct}\KeywordTok{:}\AttributeTok{ }\CharTok{true}
\end{Highlighting}
\end{Shaded}

\section{Practical Configuration
Workflows}\label{practical-configuration-workflows}

\subsection{Solo Developer Workflow}\label{solo-developer-workflow}

\subsubsection{Initial Setup}\label{initial-setup}

\begin{Shaded}
\begin{Highlighting}[]
\CommentTok{\# 1. Initialize personal configuration}
\ExtensionTok{zzcollab} \AttributeTok{{-}{-}config}\NormalTok{ init}

\CommentTok{\# 2. Set personal preferences}
\ExtensionTok{zzcollab} \AttributeTok{{-}{-}config}\NormalTok{ set team{-}name }\StringTok{"myteam"}
\ExtensionTok{zzcollab} \AttributeTok{{-}{-}config}\NormalTok{ set github{-}account }\StringTok{"myusername"}
\ExtensionTok{zzcollab} \AttributeTok{{-}{-}config}\NormalTok{ set paradigm }\StringTok{"analysis"}
\ExtensionTok{zzcollab} \AttributeTok{{-}{-}config}\NormalTok{ set build{-}mode }\StringTok{"standard"}
\ExtensionTok{zzcollab} \AttributeTok{{-}{-}config}\NormalTok{ set dotfiles{-}dir }\StringTok{"\textasciitilde{}/dotfiles"}
\end{Highlighting}
\end{Shaded}

\subsubsection{Custom Package
Configuration}\label{custom-package-configuration}

Edit \texttt{\textasciitilde{}/.zzcollab/config.yaml}:

\begin{Shaded}
\begin{Highlighting}[]
\FunctionTok{defaults}\KeywordTok{:}
\AttributeTok{  }\FunctionTok{team\_name}\KeywordTok{:}\AttributeTok{ }\StringTok{"myteam"}
\AttributeTok{  }\FunctionTok{paradigm}\KeywordTok{:}\AttributeTok{ }\StringTok{"analysis"}
\AttributeTok{  }\FunctionTok{build\_mode}\KeywordTok{:}\AttributeTok{ }\StringTok{"my\_custom"}

\FunctionTok{build\_modes}\KeywordTok{:}
\AttributeTok{  }\FunctionTok{my\_custom}\KeywordTok{:}
\AttributeTok{    }\FunctionTok{description}\KeywordTok{:}\AttributeTok{ }\StringTok{"My personal data science stack"}
\AttributeTok{    }\FunctionTok{docker\_packages}\KeywordTok{:}\AttributeTok{ }\KeywordTok{[}\StringTok{"renv"}\KeywordTok{,}\AttributeTok{ }\StringTok{"tidyverse"}\KeywordTok{,}\AttributeTok{ }\StringTok{"here"}\KeywordTok{,}\AttributeTok{ }\StringTok{"conflicted"}\KeywordTok{]}
\AttributeTok{    }\FunctionTok{renv\_packages}\KeywordTok{:}\AttributeTok{ }\KeywordTok{[}\StringTok{"renv"}\KeywordTok{,}\AttributeTok{ }\StringTok{"tidyverse"}\KeywordTok{,}\AttributeTok{ }\StringTok{"targets"}\KeywordTok{,}\AttributeTok{ }\StringTok{"pins"}\KeywordTok{,}\AttributeTok{ }\StringTok{"vetiver"}\KeywordTok{]}
\end{Highlighting}
\end{Shaded}

\subsubsection{Project Creation}\label{project-creation}

\begin{Shaded}
\begin{Highlighting}[]
\CommentTok{\# Create project using personal defaults}
\ExtensionTok{zzcollab} \AttributeTok{{-}i} \AttributeTok{{-}p}\NormalTok{ my{-}analysis}

\CommentTok{\# Customize variants for this specific project}
\BuiltInTok{cd}\NormalTok{ my{-}analysis}
\ExtensionTok{./add\_variant.sh}    \CommentTok{\# Add alpine\_minimal for CI/CD testing}
\end{Highlighting}
\end{Shaded}

\subsection{Team Leader Workflow}\label{team-leader-workflow}

\subsubsection{Team Configuration Setup}\label{team-configuration-setup}

\begin{Shaded}
\begin{Highlighting}[]
\CommentTok{\# 1. Create team project directory}
\FunctionTok{mkdir}\NormalTok{ customer{-}churn{-}analysis}
\BuiltInTok{cd}\NormalTok{ customer{-}churn{-}analysis}

\CommentTok{\# 2. Initialize with base configuration}
\ExtensionTok{zzcollab} \AttributeTok{{-}i} \AttributeTok{{-}p}\NormalTok{ customer{-}churn{-}analysis}
\end{Highlighting}
\end{Shaded}

\subsubsection{Customize Team Variants}\label{customize-team-variants}

Edit the generated \texttt{config.yaml}:

\begin{Shaded}
\begin{Highlighting}[]
\FunctionTok{team}\KeywordTok{:}
\AttributeTok{  }\FunctionTok{name}\KeywordTok{:}\AttributeTok{ }\StringTok{"datasci{-}lab"}
\AttributeTok{  }\FunctionTok{project}\KeywordTok{:}\AttributeTok{ }\StringTok{"customer{-}churn{-}analysis"}
\AttributeTok{  }\FunctionTok{description}\KeywordTok{:}\AttributeTok{ }\StringTok{"ML analysis of customer retention patterns"}

\FunctionTok{variants}\KeywordTok{:}
\CommentTok{  \# Core development environment}
\AttributeTok{  }\FunctionTok{minimal}\KeywordTok{:}
\AttributeTok{    }\FunctionTok{enabled}\KeywordTok{:}\AttributeTok{ }\CharTok{true}\CommentTok{    \# For quick testing and CI/CD}

\CommentTok{  \# Primary analysis environment}
\AttributeTok{  }\FunctionTok{analysis}\KeywordTok{:}
\AttributeTok{    }\FunctionTok{enabled}\KeywordTok{:}\AttributeTok{ }\CharTok{true}\CommentTok{    \# Main development interface}

\CommentTok{  \# Machine learning environment}
\AttributeTok{  }\FunctionTok{modeling}\KeywordTok{:}
\AttributeTok{    }\FunctionTok{enabled}\KeywordTok{:}\AttributeTok{ }\CharTok{true}\CommentTok{    \# For model development and evaluation}

\CommentTok{  \# Alpine for CI/CD}
\AttributeTok{  }\FunctionTok{alpine\_minimal}\KeywordTok{:}
\AttributeTok{    }\FunctionTok{enabled}\KeywordTok{:}\AttributeTok{ }\CharTok{true}\CommentTok{    \# Fast CI/CD testing}

\CommentTok{  \# Optional environments (disabled by default)}
\AttributeTok{  }\FunctionTok{publishing}\KeywordTok{:}
\AttributeTok{    }\FunctionTok{enabled}\KeywordTok{:}\AttributeTok{ }\CharTok{false}\CommentTok{   \# Enable when needed for reports}
\AttributeTok{  }\FunctionTok{geospatial}\KeywordTok{:}
\AttributeTok{    }\FunctionTok{enabled}\KeywordTok{:}\AttributeTok{ }\CharTok{false}\CommentTok{   \# Not needed for this project}

\FunctionTok{collaboration}\KeywordTok{:}
\AttributeTok{  }\FunctionTok{github}\KeywordTok{:}
\AttributeTok{    }\FunctionTok{auto\_create\_repo}\KeywordTok{:}\AttributeTok{ }\CharTok{true}
\AttributeTok{    }\FunctionTok{default\_visibility}\KeywordTok{:}\AttributeTok{ }\StringTok{"private"}
\AttributeTok{    }\FunctionTok{enable\_actions}\KeywordTok{:}\AttributeTok{ }\CharTok{true}

\AttributeTok{  }\FunctionTok{development}\KeywordTok{:}
\AttributeTok{    }\FunctionTok{default\_interface}\KeywordTok{:}\AttributeTok{ }\StringTok{"analysis"}
\end{Highlighting}
\end{Shaded}

\subsubsection{Build and Deploy Team
Images}\label{build-and-deploy-team-images}

\begin{Shaded}
\begin{Highlighting}[]
\CommentTok{\# Build team images with custom configuration}
\ExtensionTok{zzcollab} \AttributeTok{{-}{-}variants{-}config}\NormalTok{ config.yaml }\AttributeTok{{-}{-}github}

\CommentTok{\# Push to GitHub and Docker Hub}
\FunctionTok{git}\NormalTok{ add . }\KeywordTok{\&\&} \FunctionTok{git}\NormalTok{ commit }\AttributeTok{{-}m} \StringTok{"Initial team configuration"}
\FunctionTok{git}\NormalTok{ push origin main}
\end{Highlighting}
\end{Shaded}

\subsection{Team Member Workflow}\label{team-member-workflow}

\subsubsection{Joining an Existing Team
Project}\label{joining-an-existing-team-project}

\begin{Shaded}
\begin{Highlighting}[]
\CommentTok{\# 1. Clone team repository}
\FunctionTok{git}\NormalTok{ clone https://github.com/datasci{-}lab/customer{-}churn{-}analysis.git}
\BuiltInTok{cd}\NormalTok{ customer{-}churn{-}analysis}

\CommentTok{\# 2. Join using team\textquotesingle{}s preferred interface}
\ExtensionTok{zzcollab} \AttributeTok{{-}t}\NormalTok{ datasci{-}lab }\AttributeTok{{-}p}\NormalTok{ customer{-}churn{-}analysis }\AttributeTok{{-}I}\NormalTok{ analysis}

\CommentTok{\# 3. Start development environment}
\FunctionTok{make}\NormalTok{ docker{-}zsh}
\end{Highlighting}
\end{Shaded}

\subsubsection{Understanding Team
Configuration}\label{understanding-team-configuration}

\begin{Shaded}
\begin{Highlighting}[]
\CommentTok{\# View team\textquotesingle{}s available variants}
\ExtensionTok{zzcollab} \AttributeTok{{-}{-}config}\NormalTok{ list}

\CommentTok{\# See which Docker images are available}
\ExtensionTok{docker}\NormalTok{ images }\KeywordTok{|} \FunctionTok{grep}\NormalTok{ datasci{-}lab}

\CommentTok{\# Check team\textquotesingle{}s package configuration}
\FunctionTok{cat}\NormalTok{ config.yaml }\KeywordTok{|} \FunctionTok{grep} \AttributeTok{{-}A}\NormalTok{ 20 }\StringTok{"build:"}
\end{Highlighting}
\end{Shaded}

\subsubsection{Working with Multiple
Interfaces}\label{working-with-multiple-interfaces}

\begin{Shaded}
\begin{Highlighting}[]
\CommentTok{\# Use analysis environment for daily work}
\ExtensionTok{zzcollab} \AttributeTok{{-}t}\NormalTok{ datasci{-}lab }\AttributeTok{{-}p}\NormalTok{ customer{-}churn{-}analysis }\AttributeTok{{-}I}\NormalTok{ analysis}
\FunctionTok{make}\NormalTok{ docker{-}zsh}

\CommentTok{\# Switch to modeling environment for ML work}
\BuiltInTok{exit}  \CommentTok{\# Exit current container}
\ExtensionTok{zzcollab} \AttributeTok{{-}t}\NormalTok{ datasci{-}lab }\AttributeTok{{-}p}\NormalTok{ customer{-}churn{-}analysis }\AttributeTok{{-}I}\NormalTok{ modeling}
\FunctionTok{make}\NormalTok{ docker{-}zsh}

\CommentTok{\# Use minimal environment for testing}
\BuiltInTok{exit}
\ExtensionTok{zzcollab} \AttributeTok{{-}t}\NormalTok{ datasci{-}lab }\AttributeTok{{-}p}\NormalTok{ customer{-}churn{-}analysis }\AttributeTok{{-}I}\NormalTok{ minimal}
\FunctionTok{make}\NormalTok{ docker{-}test}
\end{Highlighting}
\end{Shaded}

\subsection{Advanced Custom Variant
Workflow}\label{advanced-custom-variant-workflow}

\subsubsection{Creating Organization-Specific
Variants}\label{creating-organization-specific-variants}

\begin{Shaded}
\begin{Highlighting}[]
\CommentTok{\# 1. Copy and customize variant library}
\FunctionTok{cp}\NormalTok{ templates/variant\_examples.yaml custom\_variants.yaml}
\end{Highlighting}
\end{Shaded}

Edit \texttt{custom\_variants.yaml} to add organization variants:

\begin{Shaded}
\begin{Highlighting}[]
\CommentTok{\# Add to existing variants}
\FunctionTok{org\_finance}\KeywordTok{:}
\AttributeTok{  }\FunctionTok{base\_image}\KeywordTok{:}\AttributeTok{ }\StringTok{"rocker/r{-}ver:latest"}
\AttributeTok{  }\FunctionTok{description}\KeywordTok{:}\AttributeTok{ }\StringTok{"Financial analysis environment with specialized packages"}
\AttributeTok{  }\FunctionTok{packages}\KeywordTok{:}
\AttributeTok{    }\KeywordTok{{-}}\AttributeTok{ }\StringTok{"renv"}
\AttributeTok{    }\KeywordTok{{-}}\AttributeTok{ }\StringTok{"devtools"}
\AttributeTok{    }\KeywordTok{{-}}\AttributeTok{ }\StringTok{"tidyverse"}
\AttributeTok{    }\KeywordTok{{-}}\AttributeTok{ }\StringTok{"tidyquant"}
\AttributeTok{    }\KeywordTok{{-}}\AttributeTok{ }\StringTok{"quantmod"}
\AttributeTok{    }\KeywordTok{{-}}\AttributeTok{ }\StringTok{"PerformanceAnalytics"}
\AttributeTok{    }\KeywordTok{{-}}\AttributeTok{ }\StringTok{"RQuantLib"}
\AttributeTok{    }\KeywordTok{{-}}\AttributeTok{ }\StringTok{"timeDate"}
\AttributeTok{  }\FunctionTok{system\_deps}\KeywordTok{:}
\AttributeTok{    }\KeywordTok{{-}}\AttributeTok{ }\StringTok{"libxml2{-}dev"}
\AttributeTok{    }\KeywordTok{{-}}\AttributeTok{ }\StringTok{"libssl{-}dev"}
\AttributeTok{    }\KeywordTok{{-}}\AttributeTok{ }\StringTok{"libcurl4{-}openssl{-}dev"}
\AttributeTok{  }\FunctionTok{category}\KeywordTok{:}\AttributeTok{ }\StringTok{"specialized"}
\AttributeTok{  }\FunctionTok{size}\KeywordTok{:}\AttributeTok{ }\StringTok{"\textasciitilde{}1.8GB"}

\FunctionTok{org\_research}\KeywordTok{:}
\AttributeTok{  }\FunctionTok{base\_image}\KeywordTok{:}\AttributeTok{ }\StringTok{"rocker/verse:latest"}
\AttributeTok{  }\FunctionTok{description}\KeywordTok{:}\AttributeTok{ }\StringTok{"Academic research environment with LaTeX and publishing tools"}
\AttributeTok{  }\FunctionTok{packages}\KeywordTok{:}
\AttributeTok{    }\KeywordTok{{-}}\AttributeTok{ }\StringTok{"renv"}
\AttributeTok{    }\KeywordTok{{-}}\AttributeTok{ }\StringTok{"devtools"}
\AttributeTok{    }\KeywordTok{{-}}\AttributeTok{ }\StringTok{"tidyverse"}
\AttributeTok{    }\KeywordTok{{-}}\AttributeTok{ }\StringTok{"rmarkdown"}
\AttributeTok{    }\KeywordTok{{-}}\AttributeTok{ }\StringTok{"bookdown"}
\AttributeTok{    }\KeywordTok{{-}}\AttributeTok{ }\StringTok{"papaja"}
\AttributeTok{    }\KeywordTok{{-}}\AttributeTok{ }\StringTok{"RefManageR"}
\AttributeTok{    }\KeywordTok{{-}}\AttributeTok{ }\StringTok{"citr"}
\AttributeTok{  }\FunctionTok{system\_deps}\KeywordTok{:}
\AttributeTok{    }\KeywordTok{{-}}\AttributeTok{ }\StringTok{"texlive{-}full"}
\AttributeTok{    }\KeywordTok{{-}}\AttributeTok{ }\StringTok{"pandoc"}
\AttributeTok{  }\FunctionTok{category}\KeywordTok{:}\AttributeTok{ }\StringTok{"specialized"}
\AttributeTok{  }\FunctionTok{size}\KeywordTok{:}\AttributeTok{ }\StringTok{"\textasciitilde{}4GB"}
\end{Highlighting}
\end{Shaded}

\subsubsection{Using Custom Variant
Library}\label{using-custom-variant-library}

Edit team \texttt{config.yaml}:

\begin{Shaded}
\begin{Highlighting}[]
\FunctionTok{build}\KeywordTok{:}
\AttributeTok{  }\FunctionTok{use\_config\_variants}\KeywordTok{:}\AttributeTok{ }\CharTok{true}
\AttributeTok{  }\FunctionTok{variant\_library}\KeywordTok{:}\AttributeTok{ }\StringTok{"custom\_variants.yaml"}\CommentTok{  \# Use custom library}

\FunctionTok{variants}\KeywordTok{:}
\AttributeTok{  }\FunctionTok{minimal}\KeywordTok{:}
\AttributeTok{    }\FunctionTok{enabled}\KeywordTok{:}\AttributeTok{ }\CharTok{true}
\AttributeTok{  }\FunctionTok{org\_finance}\KeywordTok{:}
\AttributeTok{    }\FunctionTok{enabled}\KeywordTok{:}\AttributeTok{ }\CharTok{true}\CommentTok{    \# Use custom finance variant}
\AttributeTok{  }\FunctionTok{org\_research}\KeywordTok{:}
\AttributeTok{    }\FunctionTok{enabled}\KeywordTok{:}\AttributeTok{ }\CharTok{true}\CommentTok{    \# Use custom research variant}
\end{Highlighting}
\end{Shaded}

\subsubsection{Building and Sharing Custom
Variants}\label{building-and-sharing-custom-variants}

\begin{Shaded}
\begin{Highlighting}[]
\CommentTok{\# Build with custom variant library}
\ExtensionTok{zzcollab} \AttributeTok{{-}{-}variants{-}config}\NormalTok{ config.yaml}

\CommentTok{\# Verify custom images were created}
\ExtensionTok{docker}\NormalTok{ images }\KeywordTok{|} \FunctionTok{grep}\NormalTok{ finance}
\ExtensionTok{docker}\NormalTok{ images }\KeywordTok{|} \FunctionTok{grep}\NormalTok{ research}

\CommentTok{\# Share custom variant library with organization}
\FunctionTok{git}\NormalTok{ add custom\_variants.yaml config.yaml}
\FunctionTok{git}\NormalTok{ commit }\AttributeTok{{-}m} \StringTok{"Add organization{-}specific variants"}
\FunctionTok{git}\NormalTok{ push origin main}
\end{Highlighting}
\end{Shaded}

\section{Configuration Validation and
Troubleshooting}\label{configuration-validation-and-troubleshooting}

\subsection{Validation Commands}\label{validation-commands}

\subsubsection{Comprehensive Configuration
Checking}\label{comprehensive-configuration-checking}

\begin{Shaded}
\begin{Highlighting}[]
\CommentTok{\# Check all configuration files for syntax and consistency}
\ExtensionTok{zzcollab} \AttributeTok{{-}{-}config}\NormalTok{ validate}

\CommentTok{\# List effective configuration (merged from all sources)}
\ExtensionTok{zzcollab} \AttributeTok{{-}{-}config}\NormalTok{ list}

\CommentTok{\# Check specific configuration values}
\ExtensionTok{zzcollab} \AttributeTok{{-}{-}config}\NormalTok{ get team{-}name}
\ExtensionTok{zzcollab} \AttributeTok{{-}{-}config}\NormalTok{ get variants.analysis.enabled}
\ExtensionTok{zzcollab} \AttributeTok{{-}{-}config}\NormalTok{ get build.docker.platform}
\end{Highlighting}
\end{Shaded}

\subsubsection{Variant Definition
Validation}\label{variant-definition-validation}

\begin{Shaded}
\begin{Highlighting}[]
\CommentTok{\# Validate variant\_examples.yaml syntax}
\ExtensionTok{./add\_variant.sh} \AttributeTok{{-}{-}validate}

\CommentTok{\# Check for missing or invalid variant references}
\ExtensionTok{zzcollab} \AttributeTok{{-}{-}variants{-}config}\NormalTok{ config.yaml }\AttributeTok{{-}{-}dry{-}run}

\CommentTok{\# Test variant definitions without building}
\ExtensionTok{./add\_variant.sh} \AttributeTok{{-}{-}test{-}definitions}
\end{Highlighting}
\end{Shaded}

\subsubsection{Docker and Build
Validation}\label{docker-and-build-validation}

\begin{Shaded}
\begin{Highlighting}[]
\CommentTok{\# Check Docker daemon and permissions}
\ExtensionTok{docker}\NormalTok{ info}

\CommentTok{\# Verify platform compatibility}
\ExtensionTok{docker}\NormalTok{ buildx ls}

\CommentTok{\# Test Docker build with specific variant}
\ExtensionTok{zzcollab} \AttributeTok{{-}{-}variants{-}config}\NormalTok{ config.yaml }\AttributeTok{{-}{-}build{-}only}\NormalTok{ minimal}
\end{Highlighting}
\end{Shaded}

\subsection{Common Configuration
Issues}\label{common-configuration-issues}

\subsubsection{Missing Dependencies}\label{missing-dependencies}

\textbf{Issue}: \texttt{yq} command not found

\begin{Shaded}
\begin{Highlighting}[]
\CommentTok{\# Error: YAML configuration requires yq but it\textquotesingle{}s not installed}
\ExtensionTok{Error:}\NormalTok{ Configuration validation failed }\AttributeTok{{-}}\NormalTok{ yq command not found}
\end{Highlighting}
\end{Shaded}

\textbf{Solution}: Install yq YAML processor

\begin{Shaded}
\begin{Highlighting}[]
\CommentTok{\# macOS}
\ExtensionTok{brew}\NormalTok{ install yq}

\CommentTok{\# Ubuntu/Debian}
\ExtensionTok{snap}\NormalTok{ install yq}
\CommentTok{\# or}
\FunctionTok{wget}\NormalTok{ https://github.com/mikefarah/yq/releases/latest/download/yq\_linux\_amd64 }\AttributeTok{{-}O}\NormalTok{ /usr/bin/yq}
\FunctionTok{chmod}\NormalTok{ +x /usr/bin/yq}

\CommentTok{\# Verify installation}
\ExtensionTok{yq} \AttributeTok{{-}{-}version}
\end{Highlighting}
\end{Shaded}

\subsubsection{Docker Platform Issues}\label{docker-platform-issues}

\textbf{Issue}: ARM64/AMD64 compatibility problems

\begin{Shaded}
\begin{Highlighting}[]
\CommentTok{\# Error building variant \textquotesingle{}geospatial\textquotesingle{}: platform not supported}
\ExtensionTok{Error:}\NormalTok{ rocker/geospatial:latest: no matching manifest for linux/arm64/v8}
\end{Highlighting}
\end{Shaded}

\textbf{Solution}: Configure platform compatibility

\begin{Shaded}
\begin{Highlighting}[]
\CommentTok{\# In config.yaml}
\FunctionTok{build}\KeywordTok{:}
\AttributeTok{  }\FunctionTok{docker}\KeywordTok{:}
\AttributeTok{    }\FunctionTok{platform}\KeywordTok{:}\AttributeTok{ }\StringTok{"linux/amd64"}\CommentTok{  \# Force AMD64 for compatibility}
\end{Highlighting}
\end{Shaded}

Or use ARM64-compatible variants:

\begin{Shaded}
\begin{Highlighting}[]
\FunctionTok{variants}\KeywordTok{:}
\AttributeTok{  }\FunctionTok{analysis}\KeywordTok{:}
\AttributeTok{    }\FunctionTok{enabled}\KeywordTok{:}\AttributeTok{ }\CharTok{true}\CommentTok{     \# ARM64 compatible}
\AttributeTok{  }\FunctionTok{geospatial}\KeywordTok{:}
\AttributeTok{    }\FunctionTok{enabled}\KeywordTok{:}\AttributeTok{ }\CharTok{false}\CommentTok{    \# AMD64 only {-} disable on ARM64 systems}
\AttributeTok{  }\FunctionTok{alpine\_analysis}\KeywordTok{:}
\AttributeTok{    }\FunctionTok{enabled}\KeywordTok{:}\AttributeTok{ }\CharTok{true}\CommentTok{     \# ARM64 compatible alternative}
\end{Highlighting}
\end{Shaded}

\subsubsection{Package Installation
Failures}\label{package-installation-failures}

\textbf{Issue}: Custom packages fail to install

\begin{Shaded}
\begin{Highlighting}[]
\CommentTok{\# Error during Docker build: package \textquotesingle{}obscure\_package\textquotesingle{} not available}
\ExtensionTok{Error:}\NormalTok{ installation of package }\StringTok{\textquotesingle{}obscure\_package\textquotesingle{}}\NormalTok{ had non{-}zero exit status}
\end{Highlighting}
\end{Shaded}

\textbf{Solution}: Verify package availability and dependencies

\begin{Shaded}
\begin{Highlighting}[]
\CommentTok{\# Separate core packages from experimental ones}
\FunctionTok{build\_modes}\KeywordTok{:}
\AttributeTok{  }\FunctionTok{safe\_custom}\KeywordTok{:}
\AttributeTok{    }\FunctionTok{docker\_packages}\KeywordTok{:}
\AttributeTok{      }\KeywordTok{{-}}\AttributeTok{ }\StringTok{"renv"}
\AttributeTok{      }\KeywordTok{{-}}\AttributeTok{ }\StringTok{"tidyverse"}\CommentTok{  \# Known stable packages}
\AttributeTok{    }\FunctionTok{renv\_packages}\KeywordTok{:}
\AttributeTok{      }\KeywordTok{{-}}\AttributeTok{ }\StringTok{"renv"}
\AttributeTok{      }\KeywordTok{{-}}\AttributeTok{ }\StringTok{"tidyverse"}
\AttributeTok{      }\KeywordTok{{-}}\AttributeTok{ }\StringTok{"experimental\_package"}\CommentTok{  \# Let renv handle experimental packages}
\end{Highlighting}
\end{Shaded}

\subsubsection{Permission and Access
Issues}\label{permission-and-access-issues}

\textbf{Issue}: Docker daemon permission denied

\begin{Shaded}
\begin{Highlighting}[]
\CommentTok{\# Error: permission denied while trying to connect to Docker daemon}
\ExtensionTok{Error:}\NormalTok{ Got permission denied while trying to connect to the Docker daemon socket}
\end{Highlighting}
\end{Shaded}

\textbf{Solution}: Configure Docker access

\begin{Shaded}
\begin{Highlighting}[]
\CommentTok{\# Add user to docker group (Linux)}
\FunctionTok{sudo}\NormalTok{ usermod }\AttributeTok{{-}aG}\NormalTok{ docker }\VariableTok{$USER}
\ExtensionTok{newgrp}\NormalTok{ docker}

\CommentTok{\# Restart Docker daemon (if needed)}
\FunctionTok{sudo}\NormalTok{ systemctl restart docker}

\CommentTok{\# Verify access}
\ExtensionTok{docker}\NormalTok{ run hello{-}world}
\end{Highlighting}
\end{Shaded}

\subsubsection{Configuration Hierarchy
Conflicts}\label{configuration-hierarchy-conflicts}

\textbf{Issue}: Settings not taking effect as expected

\begin{Shaded}
\begin{Highlighting}[]
\CommentTok{\# User sets paradigm = "manuscript" but projects use "analysis"}
\CommentTok{\# Project config overriding user config unexpectedly}
\end{Highlighting}
\end{Shaded}

\textbf{Solution}: Understand and verify hierarchy

\begin{Shaded}
\begin{Highlighting}[]
\CommentTok{\# Check configuration sources and precedence}
\ExtensionTok{zzcollab} \AttributeTok{{-}{-}config}\NormalTok{ list }\AttributeTok{{-}{-}verbose}

\CommentTok{\# Shows which file each setting comes from:}
\CommentTok{\# team\_name: "myteam" (from \textasciitilde{}/.zzcollab/config.yaml)}
\CommentTok{\# paradigm: "analysis" (from ./zzcollab.yaml) [OVERRIDING user config]}
\end{Highlighting}
\end{Shaded}

Edit the appropriate configuration file:

\begin{Shaded}
\begin{Highlighting}[]
\CommentTok{\# To use user default, remove from project config}
\ExtensionTok{vim}\NormalTok{ ./zzcollab.yaml  }\CommentTok{\# Remove paradigm setting}

\CommentTok{\# To set project{-}specific override, ensure it\textquotesingle{}s intentional}
\BuiltInTok{echo} \StringTok{"\# Project uses analysis paradigm instead of user default"} \OperatorTok{\textgreater{}\textgreater{}}\NormalTok{ ./zzcollab.yaml}
\end{Highlighting}
\end{Shaded}

\section{R Interface Configuration
Management}\label{r-interface-configuration-management}

ZZCOLLAB provides a complete R interface for configuration management,
allowing you to work entirely within R/RStudio environments.

\subsection{Configuration Functions}\label{configuration-functions}

\subsubsection{Basic Configuration
Management}\label{basic-configuration-management}

\begin{Shaded}
\begin{Highlighting}[]
\FunctionTok{library}\NormalTok{(zzcollab)}

\CommentTok{\# Initialize configuration}
\FunctionTok{init\_config}\NormalTok{()}

\CommentTok{\# Set configuration values}
\FunctionTok{set\_config}\NormalTok{(}\StringTok{"team\_name"}\NormalTok{, }\StringTok{"datasci{-}lab"}\NormalTok{)}
\FunctionTok{set\_config}\NormalTok{(}\StringTok{"paradigm"}\NormalTok{, }\StringTok{"analysis"}\NormalTok{)}
\FunctionTok{set\_config}\NormalTok{(}\StringTok{"build\_mode"}\NormalTok{, }\StringTok{"standard"}\NormalTok{)}
\FunctionTok{set\_config}\NormalTok{(}\StringTok{"dotfiles\_dir"}\NormalTok{, }\StringTok{"\textasciitilde{}/dotfiles"}\NormalTok{)}

\CommentTok{\# Get configuration values}
\NormalTok{team }\OtherTok{\textless{}{-}} \FunctionTok{get\_config}\NormalTok{(}\StringTok{"team\_name"}\NormalTok{)}
\NormalTok{paradigm }\OtherTok{\textless{}{-}} \FunctionTok{get\_config}\NormalTok{(}\StringTok{"paradigm"}\NormalTok{)}

\CommentTok{\# List all configuration}
\NormalTok{config\_list }\OtherTok{\textless{}{-}} \FunctionTok{list\_config}\NormalTok{()}
\FunctionTok{print}\NormalTok{(config\_list)}

\CommentTok{\# Validate configuration}
\NormalTok{validation\_result }\OtherTok{\textless{}{-}} \FunctionTok{validate\_config}\NormalTok{()}
\ControlFlowTok{if}\NormalTok{ (}\SpecialCharTok{!}\NormalTok{validation\_result}\SpecialCharTok{$}\NormalTok{valid) \{}
  \FunctionTok{message}\NormalTok{(}\StringTok{"Configuration issues found:"}\NormalTok{)}
  \FunctionTok{print}\NormalTok{(validation\_result}\SpecialCharTok{$}\NormalTok{errors)}
\NormalTok{\}}
\end{Highlighting}
\end{Shaded}

\subsubsection{Advanced Configuration with Custom
Variants}\label{advanced-configuration-with-custom-variants}

\begin{Shaded}
\begin{Highlighting}[]
\CommentTok{\# Set custom build mode with specific packages}
\FunctionTok{set\_config}\NormalTok{(}\StringTok{"build\_mode"}\NormalTok{, }\StringTok{"my\_analysis"}\NormalTok{)}

\CommentTok{\# This would correspond to a build\_modes section in your YAML:}
\CommentTok{\# build\_modes:}
\CommentTok{\#   my\_analysis:}
\CommentTok{\#     description: "Custom analysis environment"}
\CommentTok{\#     docker\_packages: ["renv", "tidyverse", "targets"]}
\CommentTok{\#     renv\_packages: ["renv", "tidyverse", "targets", "pins", "vetiver"]}

\CommentTok{\# Create project with custom configuration}
\FunctionTok{init\_project}\NormalTok{(}
  \AttributeTok{team\_name =} \StringTok{"datasci{-}lab"}\NormalTok{,}
  \AttributeTok{project\_name =} \StringTok{"customer{-}analysis"}\NormalTok{,}
  \AttributeTok{paradigm =} \StringTok{"analysis"}\NormalTok{,}
  \AttributeTok{build\_mode =} \StringTok{"my\_analysis"}\NormalTok{,}
  \AttributeTok{variants =} \FunctionTok{c}\NormalTok{(}\StringTok{"minimal"}\NormalTok{, }\StringTok{"analysis"}\NormalTok{, }\StringTok{"alpine\_minimal"}\NormalTok{)}
\NormalTok{)}
\end{Highlighting}
\end{Shaded}

\subsubsection{Configuration-Aware Project
Functions}\label{configuration-aware-project-functions}

\begin{Shaded}
\begin{Highlighting}[]
\CommentTok{\# Functions automatically use configuration defaults}
\FunctionTok{init\_project}\NormalTok{(}\StringTok{"new{-}analysis"}\NormalTok{)          }\CommentTok{\# Uses configured team\_name, paradigm, build\_mode}
\FunctionTok{join\_project}\NormalTok{(}\StringTok{"existing{-}project"}\NormalTok{)      }\CommentTok{\# Uses configured team\_name and preferences}

\CommentTok{\# Override specific settings while keeping others}
\FunctionTok{init\_project}\NormalTok{(}
  \AttributeTok{project\_name =} \StringTok{"special{-}analysis"}\NormalTok{,}
  \AttributeTok{paradigm =} \StringTok{"manuscript"}             \CommentTok{\# Override default paradigm}
  \CommentTok{\# team\_name, build\_mode from config}
\NormalTok{)}

\CommentTok{\# Explicit configuration for full control}
\FunctionTok{init\_project}\NormalTok{(}
  \AttributeTok{team\_name =} \StringTok{"special{-}team"}\NormalTok{,}
  \AttributeTok{project\_name =} \StringTok{"collaborative{-}analysis"}\NormalTok{,}
  \AttributeTok{paradigm =} \StringTok{"analysis"}\NormalTok{,}
  \AttributeTok{build\_mode =} \StringTok{"comprehensive"}\NormalTok{,}
  \AttributeTok{dotfiles\_path =} \StringTok{"\textasciitilde{}/special{-}dotfiles"}
\NormalTok{)}
\end{Highlighting}
\end{Shaded}

\subsection{Configuration Inspection and
Debugging}\label{configuration-inspection-and-debugging}

\subsubsection{Understanding Effective
Configuration}\label{understanding-effective-configuration}

\begin{Shaded}
\begin{Highlighting}[]
\CommentTok{\# See the complete effective configuration}
\NormalTok{effective\_config }\OtherTok{\textless{}{-}} \FunctionTok{get\_effective\_config}\NormalTok{()}
\FunctionTok{str}\NormalTok{(effective\_config)}

\CommentTok{\# Check configuration sources for specific values}
\NormalTok{config\_sources }\OtherTok{\textless{}{-}} \FunctionTok{get\_config\_sources}\NormalTok{()}
\FunctionTok{print}\NormalTok{(config\_sources}\SpecialCharTok{$}\NormalTok{team\_name)  }\CommentTok{\# Shows which file provides team\_name}

\CommentTok{\# Validate configuration with detailed output}
\NormalTok{validation }\OtherTok{\textless{}{-}} \FunctionTok{validate\_config}\NormalTok{(}\AttributeTok{verbose =} \ConstantTok{TRUE}\NormalTok{)}
\ControlFlowTok{if}\NormalTok{ (validation}\SpecialCharTok{$}\NormalTok{has\_warnings) \{}
  \FunctionTok{message}\NormalTok{(}\StringTok{"Configuration warnings:"}\NormalTok{)}
  \FunctionTok{sapply}\NormalTok{(validation}\SpecialCharTok{$}\NormalTok{warnings, message)}
\NormalTok{\}}
\end{Highlighting}
\end{Shaded}

\subsubsection{Working with Multiple
Configurations}\label{working-with-multiple-configurations}

\begin{Shaded}
\begin{Highlighting}[]
\CommentTok{\# Save current configuration state}
\NormalTok{current\_config }\OtherTok{\textless{}{-}} \FunctionTok{save\_config\_state}\NormalTok{()}

\CommentTok{\# Temporarily change configuration for specific project}
\NormalTok{temp\_config }\OtherTok{\textless{}{-}} \FunctionTok{list}\NormalTok{(}
  \AttributeTok{team\_name =} \StringTok{"temporary{-}team"}\NormalTok{,}
  \AttributeTok{paradigm =} \StringTok{"package"}\NormalTok{,}
  \AttributeTok{build\_mode =} \StringTok{"fast"}
\NormalTok{)}
\FunctionTok{apply\_temp\_config}\NormalTok{(temp\_config)}

\CommentTok{\# Do work with temporary configuration}
\FunctionTok{init\_project}\NormalTok{(}\StringTok{"temp{-}package{-}project"}\NormalTok{)}

\CommentTok{\# Restore original configuration}
\FunctionTok{restore\_config\_state}\NormalTok{(current\_config)}

\CommentTok{\# Verify restoration}
\FunctionTok{get\_config}\NormalTok{(}\StringTok{"team\_name"}\NormalTok{)  }\CommentTok{\# Should be back to original value}
\end{Highlighting}
\end{Shaded}

\section{Best Practices and
Recommendations}\label{best-practices-and-recommendations}

\subsection{Configuration Strategy by Use
Case}\label{configuration-strategy-by-use-case}

\subsubsection{Solo Researcher}\label{solo-researcher}

\begin{itemize}
\tightlist
\item
  \textbf{User config only}: Set personal defaults in
  \texttt{\textasciitilde{}/.zzcollab/config.yaml}
\item
  \textbf{Minimal variants}: Use 2-3 variants (minimal, analysis,
  alpine\_minimal)
\item
  \textbf{Standard build mode}: Balance between functionality and build
  time
\item
  \textbf{Simple package lists}: Stick to well-established packages
\end{itemize}

\begin{Shaded}
\begin{Highlighting}[]
\CommentTok{\# \textasciitilde{}/.zzcollab/config.yaml for solo researcher}
\FunctionTok{defaults}\KeywordTok{:}
\AttributeTok{  }\FunctionTok{team\_name}\KeywordTok{:}\AttributeTok{ }\StringTok{"myresearch"}
\AttributeTok{  }\FunctionTok{paradigm}\KeywordTok{:}\AttributeTok{ }\StringTok{"analysis"}
\AttributeTok{  }\FunctionTok{build\_mode}\KeywordTok{:}\AttributeTok{ }\StringTok{"standard"}

\CommentTok{\# No custom build\_modes {-} use built{-}in standard mode}
\end{Highlighting}
\end{Shaded}

\subsubsection{Small Research Team (2-5
people)}\label{small-research-team-2-5-people}

\begin{itemize}
\tightlist
\item
  \textbf{Project configs}: Share configuration via
  \texttt{./zzcollab.yaml} in repositories
\item
  \textbf{Focused variants}: 3-5 variants covering team's research
  domains
\item
  \textbf{Paradigm alignment}: Team agrees on primary paradigm
  (analysis/manuscript)
\item
  \textbf{Package coordination}: Custom build modes for shared analysis
  workflows
\end{itemize}

\begin{Shaded}
\begin{Highlighting}[]
\CommentTok{\# ./zzcollab.yaml for small team}
\FunctionTok{team}\KeywordTok{:}
\AttributeTok{  }\FunctionTok{name}\KeywordTok{:}\AttributeTok{ }\StringTok{"psychlab"}
\AttributeTok{  }\FunctionTok{project}\KeywordTok{:}\AttributeTok{ }\StringTok{"attention{-}study"}

\FunctionTok{variants}\KeywordTok{:}
\AttributeTok{  }\FunctionTok{minimal}\KeywordTok{:}
\AttributeTok{    }\FunctionTok{enabled}\KeywordTok{:}\AttributeTok{ }\CharTok{true}\CommentTok{    \# Quick testing}
\AttributeTok{  }\FunctionTok{analysis}\KeywordTok{:}
\AttributeTok{    }\FunctionTok{enabled}\KeywordTok{:}\AttributeTok{ }\CharTok{true}\CommentTok{    \# Primary analysis}
\AttributeTok{  }\FunctionTok{publishing}\KeywordTok{:}
\AttributeTok{    }\FunctionTok{enabled}\KeywordTok{:}\AttributeTok{ }\CharTok{true}\CommentTok{    \# Manuscript preparation}

\FunctionTok{build\_modes}\KeywordTok{:}
\AttributeTok{  }\FunctionTok{psych\_analysis}\KeywordTok{:}
\AttributeTok{    }\FunctionTok{description}\KeywordTok{:}\AttributeTok{ }\StringTok{"Psychology research workflow"}
\AttributeTok{    }\FunctionTok{docker\_packages}\KeywordTok{:}\AttributeTok{ }\KeywordTok{[}\StringTok{"renv"}\KeywordTok{,}\AttributeTok{ }\StringTok{"tidyverse"}\KeywordTok{,}\AttributeTok{ }\StringTok{"here"}\KeywordTok{]}
\AttributeTok{    }\FunctionTok{renv\_packages}\KeywordTok{:}\AttributeTok{ }\KeywordTok{[}\StringTok{"renv"}\KeywordTok{,}\AttributeTok{ }\StringTok{"tidyverse"}\KeywordTok{,}\AttributeTok{ }\StringTok{"psych"}\KeywordTok{,}\AttributeTok{ }\StringTok{"lavaan"}\KeywordTok{,}\AttributeTok{ }\StringTok{"sem"}\KeywordTok{,}\AttributeTok{ }\StringTok{"papaja"}\KeywordTok{]}
\end{Highlighting}
\end{Shaded}

\subsubsection{Large Research Organization (10+
people)}\label{large-research-organization-10-people}

\begin{itemize}
\tightlist
\item
  \textbf{System configs}: Organization defaults in
  \texttt{/etc/zzcollab/config.yaml}
\item
  \textbf{Extensive variants}: Full variant library available to teams
\item
  \textbf{Multiple paradigms}: Support analysis, manuscript, and package
  paradigms
\item
  \textbf{Custom variant library}: Organization-specific variants in
  custom libraries
\end{itemize}

\begin{Shaded}
\begin{Highlighting}[]
\CommentTok{\# /etc/zzcollab/config.yaml for organization}
\FunctionTok{defaults}\KeywordTok{:}
\AttributeTok{  }\FunctionTok{github\_account}\KeywordTok{:}\AttributeTok{ }\StringTok{"university{-}research"}

\CommentTok{\# Custom organization variants}
\FunctionTok{build}\KeywordTok{:}
\AttributeTok{  }\FunctionTok{variant\_library}\KeywordTok{:}\AttributeTok{ }\StringTok{"/etc/zzcollab/org\_variants.yaml"}

\FunctionTok{collaboration}\KeywordTok{:}
\AttributeTok{  }\FunctionTok{github}\KeywordTok{:}
\AttributeTok{    }\FunctionTok{default\_visibility}\KeywordTok{:}\AttributeTok{ }\StringTok{"private"}
\AttributeTok{    }\FunctionTok{enable\_actions}\KeywordTok{:}\AttributeTok{ }\CharTok{true}

\AttributeTok{  }\FunctionTok{documentation}\KeywordTok{:}
\AttributeTok{    }\FunctionTok{auto\_generate\_readme}\KeywordTok{:}\AttributeTok{ }\CharTok{true}
\AttributeTok{    }\FunctionTok{contributing\_guide}\KeywordTok{:}\AttributeTok{ }\CharTok{true}
\end{Highlighting}
\end{Shaded}

\subsection{Package Management Best
Practices}\label{package-management-best-practices}

\subsubsection{Docker vs renv Package
Separation}\label{docker-vs-renv-package-separation}

\textbf{Docker Packages} (installed during image build): - System-wide
development tools: \texttt{renv}, \texttt{devtools}, \texttt{here},
\texttt{conflicted} - Essential utilities everyone needs:
\texttt{usethis}, \texttt{testthat}, \texttt{knitr} - Large, stable
dependencies: \texttt{tidyverse} (if used by most team members)

\textbf{renv Packages} (project-specific): - Analysis-specific packages:
domain-specific statistical packages - Experimental or cutting-edge
packages: newly released packages - Version-sensitive packages: packages
requiring specific versions - Data-specific packages: packages for
particular datasets or APIs

\begin{Shaded}
\begin{Highlighting}[]
\CommentTok{\# Good separation example}
\FunctionTok{build\_modes}\KeywordTok{:}
\AttributeTok{  }\FunctionTok{good\_separation}\KeywordTok{:}
\AttributeTok{    }\FunctionTok{docker\_packages}\KeywordTok{:}
\AttributeTok{      }\KeywordTok{{-}}\AttributeTok{ }\StringTok{"renv"}\CommentTok{          \# Package management (everyone needs)}
\AttributeTok{      }\KeywordTok{{-}}\AttributeTok{ }\StringTok{"devtools"}\CommentTok{      \# Development tools (everyone needs)}
\AttributeTok{      }\KeywordTok{{-}}\AttributeTok{ }\StringTok{"here"}\CommentTok{          \# Path management (everyone needs)}
\AttributeTok{      }\KeywordTok{{-}}\AttributeTok{ }\StringTok{"conflicted"}\CommentTok{    \# Namespace management (everyone needs)}
\AttributeTok{    }\FunctionTok{renv\_packages}\KeywordTok{:}
\AttributeTok{      }\KeywordTok{{-}}\AttributeTok{ }\StringTok{"renv"}
\AttributeTok{      }\KeywordTok{{-}}\AttributeTok{ }\StringTok{"tidyverse"}\CommentTok{     \# Analysis packages (project{-}specific)}
\AttributeTok{      }\KeywordTok{{-}}\AttributeTok{ }\StringTok{"targets"}\CommentTok{       \# Workflow management (project{-}specific)}
\AttributeTok{      }\KeywordTok{{-}}\AttributeTok{ }\StringTok{"specific\_analysis\_package"}\CommentTok{  \# Domain{-}specific (project{-}specific)}
\end{Highlighting}
\end{Shaded}

\subsubsection{Version Management
Strategy}\label{version-management-strategy}

\begin{Shaded}
\begin{Highlighting}[]
\CommentTok{\# Pin critical package versions in project configs}
\FunctionTok{build\_modes}\KeywordTok{:}
\AttributeTok{  }\FunctionTok{stable\_analysis}\KeywordTok{:}
\AttributeTok{    }\FunctionTok{description}\KeywordTok{:}\AttributeTok{ }\StringTok{"Stable versions for reproducible analysis"}
\AttributeTok{    }\FunctionTok{docker\_packages}\KeywordTok{:}\AttributeTok{ }\KeywordTok{[}\StringTok{"renv"}\KeywordTok{,}\AttributeTok{ }\StringTok{"devtools"}\KeywordTok{,}\AttributeTok{ }\StringTok{"here"}\KeywordTok{]}
\AttributeTok{    }\FunctionTok{renv\_packages}\KeywordTok{:}
\AttributeTok{      }\KeywordTok{{-}}\AttributeTok{ }\StringTok{"renv"}
\AttributeTok{      }\KeywordTok{{-}}\AttributeTok{ }\StringTok{"dplyr@1.1.0"}\CommentTok{        \# Pin specific version}
\AttributeTok{      }\KeywordTok{{-}}\AttributeTok{ }\StringTok{"ggplot2@3.4.0"}\CommentTok{      \# Pin specific version}
\AttributeTok{      }\KeywordTok{{-}}\AttributeTok{ }\StringTok{"latest\_package"}\CommentTok{     \# Use latest for new packages}
\end{Highlighting}
\end{Shaded}

\subsection{Performance Optimization}\label{performance-optimization}

\subsubsection{Build Time Optimization}\label{build-time-optimization}

\begin{enumerate}
\def\labelenumi{\arabic{enumi}.}
\tightlist
\item
  \textbf{Use appropriate build modes}: Fast mode for CI/CD,
  Comprehensive for production
\item
  \textbf{Minimize Docker packages}: Only install widely-used packages
  in Docker
\item
  \textbf{Layer caching}: Order packages by stability (stable packages
  first)
\item
  \textbf{Parallel builds}: Enable parallel variant building in
  configuration
\end{enumerate}

\begin{Shaded}
\begin{Highlighting}[]
\CommentTok{\# Optimized for build performance}
\FunctionTok{build}\KeywordTok{:}
\AttributeTok{  }\FunctionTok{docker}\KeywordTok{:}
\AttributeTok{    }\FunctionTok{parallel\_builds}\KeywordTok{:}\AttributeTok{ }\CharTok{true}
\AttributeTok{    }\FunctionTok{memory\_limit}\KeywordTok{:}\AttributeTok{ }\StringTok{"4g"}

\AttributeTok{  }\FunctionTok{packages}\KeywordTok{:}
\AttributeTok{    }\FunctionTok{parallel\_installs}\KeywordTok{:}\AttributeTok{ }\CharTok{true}
\end{Highlighting}
\end{Shaded}

\subsubsection{Container Size
Optimization}\label{container-size-optimization}

\begin{enumerate}
\def\labelenumi{\arabic{enumi}.}
\tightlist
\item
  \textbf{Choose appropriate base images}: Use minimal bases when
  possible
\item
  \textbf{Use Alpine variants}: For CI/CD and testing workflows
\item
  \textbf{Separate concerns}: Use different variants for different
  purposes
\end{enumerate}

\begin{Shaded}
\begin{Highlighting}[]
\CommentTok{\# Size{-}optimized variant strategy}
\FunctionTok{variants}\KeywordTok{:}
\AttributeTok{  }\FunctionTok{alpine\_minimal}\KeywordTok{:}
\AttributeTok{    }\FunctionTok{enabled}\KeywordTok{:}\AttributeTok{ }\CharTok{true}\CommentTok{     \# \textasciitilde{}200MB for CI/CD}
\AttributeTok{  }\FunctionTok{analysis}\KeywordTok{:}
\AttributeTok{    }\FunctionTok{enabled}\KeywordTok{:}\AttributeTok{ }\CharTok{true}\CommentTok{     \# \textasciitilde{}1.2GB for development}
\CommentTok{  \# Don\textquotesingle{}t enable comprehensive variants unless needed}
\end{Highlighting}
\end{Shaded}

\subsection{Security and Access
Management}\label{security-and-access-management}

\subsubsection{GitHub Repository
Settings}\label{github-repository-settings}

\begin{Shaded}
\begin{Highlighting}[]
\FunctionTok{collaboration}\KeywordTok{:}
\AttributeTok{  }\FunctionTok{github}\KeywordTok{:}
\AttributeTok{    }\FunctionTok{default\_visibility}\KeywordTok{:}\AttributeTok{ }\StringTok{"private"}\CommentTok{    \# Default to private repositories}
\AttributeTok{    }\FunctionTok{branch\_protection}\KeywordTok{:}\AttributeTok{ }\CharTok{true}\CommentTok{          \# Require PR reviews}

\AttributeTok{    }\FunctionTok{repository}\KeywordTok{:}
\AttributeTok{      }\FunctionTok{allow\_merge\_commits}\KeywordTok{:}\AttributeTok{ }\CharTok{false}\CommentTok{     \# Require clean history}
\AttributeTok{      }\FunctionTok{allow\_squash\_merging}\KeywordTok{:}\AttributeTok{ }\CharTok{true}\CommentTok{     \# Allow squash merging}
\AttributeTok{      }\FunctionTok{delete\_head\_branches}\KeywordTok{:}\AttributeTok{ }\CharTok{true}\CommentTok{     \# Clean up after merging}
\end{Highlighting}
\end{Shaded}

\subsubsection{Container Security}\label{container-security}

\begin{Shaded}
\begin{Highlighting}[]
\FunctionTok{development}\KeywordTok{:}
\AttributeTok{  }\FunctionTok{container}\KeywordTok{:}
\AttributeTok{    }\FunctionTok{default\_user}\KeywordTok{:}\AttributeTok{ }\StringTok{"analyst"}\CommentTok{          \# Don\textquotesingle{}t run as root}
\AttributeTok{    }\FunctionTok{read\_only\_volumes}\KeywordTok{:}\AttributeTok{ }\CharTok{true}\CommentTok{          \# Mount shared volumes read{-}only when possible}
\end{Highlighting}
\end{Shaded}

\subsubsection{Access Control}\label{access-control}

\begin{itemize}
\tightlist
\item
  \textbf{User configs}: Personal settings only (team names,
  preferences)
\item
  \textbf{Project configs}: Team settings (variants, collaboration
  rules)
\item
  \textbf{System configs}: Organization policies (security, defaults)
\end{itemize}

This separation ensures that sensitive organization settings cannot be
overridden by individual users or projects.

\section{Conclusion}\label{conclusion}

ZZCOLLAB's configuration system provides powerful, flexible control over
research environments while maintaining simplicity for common use cases.
The multi-layered hierarchy allows individual preferences, team
coordination, and organizational policies to coexist seamlessly.

Key takeaways:

\begin{enumerate}
\def\labelenumi{\arabic{enumi}.}
\tightlist
\item
  \textbf{Start simple}: Use built-in build modes and standard variants
  for most projects
\item
  \textbf{Customize gradually}: Add custom variants and packages as
  needs become clear
\item
  \textbf{Coordinate with teams}: Use project configs to ensure team
  consistency
\item
  \textbf{Optimize for workflow}: Choose variants and packages that
  match your research paradigm
\item
  \textbf{Validate regularly}: Use configuration validation to catch
  issues early
\end{enumerate}

The configuration system scales from solo researchers to large
organizations, providing reproducible, customizable research
environments that eliminate configuration drift and enable seamless
collaboration.

For detailed technical documentation, see: - \texttt{zzcollab\ -\/-help}
for command-line options - \texttt{help(package\ =\ "zzcollab")} for R
interface documentation - \texttt{templates/variant\_examples.yaml} for
available variants - \texttt{PARADIGM\_GUIDE.md} for paradigm-specific
guidance

\end{document}
