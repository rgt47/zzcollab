% Options for packages loaded elsewhere
\PassOptionsToPackage{unicode}{hyperref}
\PassOptionsToPackage{hyphens}{url}
\documentclass[
]{article}
\usepackage{xcolor}
\usepackage[margin=1in]{geometry}
\usepackage{amsmath,amssymb}
\setcounter{secnumdepth}{-\maxdimen} % remove section numbering
\usepackage{iftex}
\ifPDFTeX
  \usepackage[T1]{fontenc}
  \usepackage[utf8]{inputenc}
  \usepackage{textcomp} % provide euro and other symbols
\else % if luatex or xetex
  \usepackage{unicode-math} % this also loads fontspec
  \defaultfontfeatures{Scale=MatchLowercase}
  \defaultfontfeatures[\rmfamily]{Ligatures=TeX,Scale=1}
\fi
\usepackage{lmodern}
\ifPDFTeX\else
  % xetex/luatex font selection
\fi
% Use upquote if available, for straight quotes in verbatim environments
\IfFileExists{upquote.sty}{\usepackage{upquote}}{}
\IfFileExists{microtype.sty}{% use microtype if available
  \usepackage[]{microtype}
  \UseMicrotypeSet[protrusion]{basicmath} % disable protrusion for tt fonts
}{}
\makeatletter
\@ifundefined{KOMAClassName}{% if non-KOMA class
  \IfFileExists{parskip.sty}{%
    \usepackage{parskip}
  }{% else
    \setlength{\parindent}{0pt}
    \setlength{\parskip}{6pt plus 2pt minus 1pt}}
}{% if KOMA class
  \KOMAoptions{parskip=half}}
\makeatother
\usepackage{color}
\usepackage{fancyvrb}
\newcommand{\VerbBar}{|}
\newcommand{\VERB}{\Verb[commandchars=\\\{\}]}
\DefineVerbatimEnvironment{Highlighting}{Verbatim}{commandchars=\\\{\}}
% Add ',fontsize=\small' for more characters per line
\usepackage{framed}
\definecolor{shadecolor}{RGB}{248,248,248}
\newenvironment{Shaded}{\begin{snugshade}}{\end{snugshade}}
\newcommand{\AlertTok}[1]{\textcolor[rgb]{0.94,0.16,0.16}{#1}}
\newcommand{\AnnotationTok}[1]{\textcolor[rgb]{0.56,0.35,0.01}{\textbf{\textit{#1}}}}
\newcommand{\AttributeTok}[1]{\textcolor[rgb]{0.13,0.29,0.53}{#1}}
\newcommand{\BaseNTok}[1]{\textcolor[rgb]{0.00,0.00,0.81}{#1}}
\newcommand{\BuiltInTok}[1]{#1}
\newcommand{\CharTok}[1]{\textcolor[rgb]{0.31,0.60,0.02}{#1}}
\newcommand{\CommentTok}[1]{\textcolor[rgb]{0.56,0.35,0.01}{\textit{#1}}}
\newcommand{\CommentVarTok}[1]{\textcolor[rgb]{0.56,0.35,0.01}{\textbf{\textit{#1}}}}
\newcommand{\ConstantTok}[1]{\textcolor[rgb]{0.56,0.35,0.01}{#1}}
\newcommand{\ControlFlowTok}[1]{\textcolor[rgb]{0.13,0.29,0.53}{\textbf{#1}}}
\newcommand{\DataTypeTok}[1]{\textcolor[rgb]{0.13,0.29,0.53}{#1}}
\newcommand{\DecValTok}[1]{\textcolor[rgb]{0.00,0.00,0.81}{#1}}
\newcommand{\DocumentationTok}[1]{\textcolor[rgb]{0.56,0.35,0.01}{\textbf{\textit{#1}}}}
\newcommand{\ErrorTok}[1]{\textcolor[rgb]{0.64,0.00,0.00}{\textbf{#1}}}
\newcommand{\ExtensionTok}[1]{#1}
\newcommand{\FloatTok}[1]{\textcolor[rgb]{0.00,0.00,0.81}{#1}}
\newcommand{\FunctionTok}[1]{\textcolor[rgb]{0.13,0.29,0.53}{\textbf{#1}}}
\newcommand{\ImportTok}[1]{#1}
\newcommand{\InformationTok}[1]{\textcolor[rgb]{0.56,0.35,0.01}{\textbf{\textit{#1}}}}
\newcommand{\KeywordTok}[1]{\textcolor[rgb]{0.13,0.29,0.53}{\textbf{#1}}}
\newcommand{\NormalTok}[1]{#1}
\newcommand{\OperatorTok}[1]{\textcolor[rgb]{0.81,0.36,0.00}{\textbf{#1}}}
\newcommand{\OtherTok}[1]{\textcolor[rgb]{0.56,0.35,0.01}{#1}}
\newcommand{\PreprocessorTok}[1]{\textcolor[rgb]{0.56,0.35,0.01}{\textit{#1}}}
\newcommand{\RegionMarkerTok}[1]{#1}
\newcommand{\SpecialCharTok}[1]{\textcolor[rgb]{0.81,0.36,0.00}{\textbf{#1}}}
\newcommand{\SpecialStringTok}[1]{\textcolor[rgb]{0.31,0.60,0.02}{#1}}
\newcommand{\StringTok}[1]{\textcolor[rgb]{0.31,0.60,0.02}{#1}}
\newcommand{\VariableTok}[1]{\textcolor[rgb]{0.00,0.00,0.00}{#1}}
\newcommand{\VerbatimStringTok}[1]{\textcolor[rgb]{0.31,0.60,0.02}{#1}}
\newcommand{\WarningTok}[1]{\textcolor[rgb]{0.56,0.35,0.01}{\textbf{\textit{#1}}}}
\usepackage{graphicx}
\makeatletter
\newsavebox\pandoc@box
\newcommand*\pandocbounded[1]{% scales image to fit in text height/width
  \sbox\pandoc@box{#1}%
  \Gscale@div\@tempa{\textheight}{\dimexpr\ht\pandoc@box+\dp\pandoc@box\relax}%
  \Gscale@div\@tempb{\linewidth}{\wd\pandoc@box}%
  \ifdim\@tempb\p@<\@tempa\p@\let\@tempa\@tempb\fi% select the smaller of both
  \ifdim\@tempa\p@<\p@\scalebox{\@tempa}{\usebox\pandoc@box}%
  \else\usebox{\pandoc@box}%
  \fi%
}
% Set default figure placement to htbp
\def\fps@figure{htbp}
\makeatother
\setlength{\emergencystretch}{3em} % prevent overfull lines
\providecommand{\tightlist}{%
  \setlength{\itemsep}{0pt}\setlength{\parskip}{0pt}}
\usepackage{bookmark}
\IfFileExists{xurl.sty}{\usepackage{xurl}}{} % add URL line breaks if available
\urlstyle{same}
\hypersetup{
  pdftitle={Solo Developer Workflow Guide},
  pdfauthor={ZZCOLLAB Team},
  hidelinks,
  pdfcreator={LaTeX via pandoc}}

\title{Solo Developer Workflow Guide}
\author{ZZCOLLAB Team}
\date{2025-09-25}

\begin{document}
\maketitle

\section{ZZCOLLAB Solo Developer Workflow
Guide}\label{zzcollab-solo-developer-workflow-guide}

This vignette describes the implementation of reproducible computational
research environments for individual researchers using the ZZCOLLAB
framework.

\subsection{Initial Setup (One-Time)}\label{initial-setup-one-time}

\subsubsection{1. Install ZZCOLLAB
System}\label{install-zzcollab-system}

\begin{Shaded}
\begin{Highlighting}[]
\CommentTok{\# Clone and install zzcollab}
\FunctionTok{git}\NormalTok{ clone https://github.com/rgt47/zzcollab.git}
\BuiltInTok{cd}\NormalTok{ zzcollab }\KeywordTok{\&\&} \ExtensionTok{./install.sh}

\CommentTok{\# Verify installation  }
\ExtensionTok{zzcollab} \AttributeTok{{-}{-}help} \KeywordTok{\&\&} \FunctionTok{which}\NormalTok{ zzcollab}
\end{Highlighting}
\end{Shaded}

\subsubsection{2. Configuration Setup
(Recommended)}\label{configuration-setup-recommended}

ZZCOLLAB includes a configuration system that reduces redundant
parameter specification by establishing project defaults.

\begin{Shaded}
\begin{Highlighting}[]
\CommentTok{\# Initialize configuration file}
\ExtensionTok{zzcollab} \AttributeTok{{-}{-}config}\NormalTok{ init}

\CommentTok{\# Set your defaults (customize as needed)}
\ExtensionTok{zzcollab} \AttributeTok{{-}{-}config}\NormalTok{ set team{-}name }\StringTok{"rgt47"}              \CommentTok{\# Your Docker Hub account}
\ExtensionTok{zzcollab} \AttributeTok{{-}{-}config}\NormalTok{ set github{-}account }\StringTok{"rgt47"}        \CommentTok{\# Your GitHub username  }
\ExtensionTok{zzcollab} \AttributeTok{{-}{-}config}\NormalTok{ set build{-}mode }\StringTok{"standard"}         \CommentTok{\# fast, standard, comprehensive}
\ExtensionTok{zzcollab} \AttributeTok{{-}{-}config}\NormalTok{ set dotfiles{-}dir }\StringTok{"\textasciitilde{}/dotfiles"}     \CommentTok{\# Path to your dotfiles}

\CommentTok{\# View your configuration}
\ExtensionTok{zzcollab} \AttributeTok{{-}{-}config}\NormalTok{ list}
\end{Highlighting}
\end{Shaded}

\subsection{Project Creation}\label{project-creation}

\subsubsection{Choose Your Development
Environment}\label{choose-your-development-environment}

\textbf{Standard Implementation:}

\begin{Shaded}
\begin{Highlighting}[]
\CommentTok{\# Creates default variants and GitHub repository}
\ExtensionTok{zzcollab} \AttributeTok{{-}i} \AttributeTok{{-}p}\NormalTok{ myproject }\AttributeTok{{-}{-}github}    \CommentTok{\# Creates: minimal + analysis variants}
\end{Highlighting}
\end{Shaded}

\textbf{Advanced Variant Selection:}

\begin{Shaded}
\begin{Highlighting}[]
\FunctionTok{mkdir}\NormalTok{ myproject }\KeywordTok{\&\&} \BuiltInTok{cd}\NormalTok{ myproject}
\ExtensionTok{zzcollab} \AttributeTok{{-}i} \AttributeTok{{-}p}\NormalTok{ myproject             }\CommentTok{\# Creates project + config.yaml}
\ExtensionTok{./add\_variant.sh}                     \CommentTok{\# Browse comprehensive variant library}

\CommentTok{\# Interactive menu shows 14 variants:}
\CommentTok{\# 📦 STANDARD RESEARCH ENVIRONMENTS}
\CommentTok{\#  1) minimal          \textasciitilde{}800MB  {-} Essential R packages only}
\CommentTok{\#  2) analysis         \textasciitilde{}1.2GB  {-} Tidyverse + data analysis tools  }
\CommentTok{\#  3) modeling         \textasciitilde{}1.5GB  {-} Machine learning with tidymodels}
\CommentTok{\#  4) publishing       \textasciitilde{}3GB    {-} LaTeX, Quarto, bookdown, blogdown}
\CommentTok{\#  5) shiny            \textasciitilde{}1.8GB  {-} Interactive web applications}
\CommentTok{\#  6) shiny\_verse      \textasciitilde{}3.5GB  {-} Shiny with tidyverse + publishing}
\CommentTok{\#}
\CommentTok{\# 🔬 SPECIALIZED DOMAINS}
\CommentTok{\#  7) bioinformatics   \textasciitilde{}2GB    {-} Bioconductor genomics packages}
\CommentTok{\#  8) geospatial       \textasciitilde{}2.5GB  {-} sf, terra, leaflet mapping tools}
\CommentTok{\#}
\CommentTok{\# 🏔️ LIGHTWEIGHT ALPINE VARIANTS  }
\CommentTok{\#  9) alpine\_minimal   \textasciitilde{}200MB  {-} Ultra{-}lightweight for CI/CD}
\CommentTok{\# 10) alpine\_analysis  \textasciitilde{}400MB  {-} Essential analysis in tiny container}
\CommentTok{\# 11) hpc\_alpine       \textasciitilde{}600MB  {-} High{-}performance parallel processing}
\CommentTok{\#}
\CommentTok{\# 🧪 R{-}HUB }\AlertTok{TESTING}\CommentTok{ ENVIRONMENTS}
\CommentTok{\# 12) rhub\_ubuntu      \textasciitilde{}1GB    {-} CRAN{-}compatible package testing}
\CommentTok{\# 13) rhub\_fedora      \textasciitilde{}1.2GB  {-} Test against R{-}devel}
\CommentTok{\# 14) rhub\_windows     \textasciitilde{}1.5GB  {-} Windows compatibility testing}

\CommentTok{\# Select variants that match your workflow, then:}
\ExtensionTok{zzcollab} \AttributeTok{{-}{-}variants{-}config}\NormalTok{ config.yaml }\AttributeTok{{-}{-}github}
\end{Highlighting}
\end{Shaded}

\textbf{Traditional Implementation (Three-variant limit):}

\begin{Shaded}
\begin{Highlighting}[]
\CommentTok{\# Traditional: Limited to shell/rstudio/verse only}
\ExtensionTok{zzcollab} \AttributeTok{{-}i} \AttributeTok{{-}p}\NormalTok{ myproject }\AttributeTok{{-}B}\NormalTok{ rstudio }\AttributeTok{{-}{-}github}    \CommentTok{\# RStudio interface}
\ExtensionTok{zzcollab} \AttributeTok{{-}i} \AttributeTok{{-}p}\NormalTok{ myproject }\AttributeTok{{-}B}\NormalTok{ all }\AttributeTok{{-}{-}github}        \CommentTok{\# All 3 legacy variants}
\end{Highlighting}
\end{Shaded}

\subsubsection{Suggested Variant Combinations by Research
Domain}\label{suggested-variant-combinations-by-research-domain}

\begin{itemize}
\tightlist
\item
  \textbf{Statistical Analysis}: \texttt{analysis} + \texttt{publishing}
  (tidyverse + reporting)
\item
  \textbf{Bioinformatics}: \texttt{bioinformatics} +
  \texttt{alpine\_minimal} (domain packages + testing)
\item
  \textbf{Package Development}: \texttt{minimal} + \texttt{rhub\_ubuntu}
  (development + validation)
\item
  \textbf{Interactive Applications}: \texttt{shiny\_verse} +
  \texttt{alpine\_minimal} (web development + deployment)
\item
  \textbf{Quantitative Research}: \texttt{modeling} +
  \texttt{publishing} (statistical modeling + manuscript preparation)
\end{itemize}

\subsection{Daily Development
Workflow}\label{daily-development-workflow}

\subsubsection{1. Start Development
Environment}\label{start-development-environment}

\begin{Shaded}
\begin{Highlighting}[]
\BuiltInTok{cd}\NormalTok{ myproject}

\CommentTok{\# Choose your interface based on selected variants:}
\FunctionTok{make}\NormalTok{ docker{-}zsh         }\CommentTok{\# Shell interface (works with any variant)}
\FunctionTok{make}\NormalTok{ docker{-}rstudio     }\CommentTok{\# RStudio Server at localhost:8787 (if rstudio/shiny variants)}
\FunctionTok{make}\NormalTok{ docker{-}r           }\CommentTok{\# R console only}
\FunctionTok{make}\NormalTok{ docker{-}verse       }\CommentTok{\# Publishing workflow with LaTeX (if publishing variant)}
\end{Highlighting}
\end{Shaded}

\subsubsection{2. Iterative Development (Inside
Container)}\label{iterative-development-inside-container}

\textbf{Working in the container:}

\begin{Shaded}
\begin{Highlighting}[]
\CommentTok{\# You\textquotesingle{}re now inside the Docker container with all packages pre{-}installed}
\CommentTok{\# Your project directory is mounted at /home/analyst/project}

\CommentTok{\# R package development}
\ExtensionTok{R}
\FunctionTok{devtools::load\_all()}           \CommentTok{\# Load your package functions}
\FunctionTok{devtools::test()}              \CommentTok{\# Run tests}
\FunctionTok{devtools::document()}          \CommentTok{\# Generate documentation}
\FunctionTok{quit()}

\CommentTok{\# Analysis scripts}
\ExtensionTok{vim}\NormalTok{ scripts/01\_data\_analysis.R    }\CommentTok{\# Create/edit analysis}
\ExtensionTok{R} \AttributeTok{{-}{-}vanilla} \OperatorTok{\textless{}}\NormalTok{ scripts/01\_data\_analysis.R  }\CommentTok{\# Run script}

\CommentTok{\# Create new functions}
\ExtensionTok{vim}\NormalTok{ R/my\_functions.R          }\CommentTok{\# Add R functions}
\ExtensionTok{vim}\NormalTok{ tests/testthat/test{-}my\_functions.R  }\CommentTok{\# Write tests}

\CommentTok{\# Install additional packages (will be tracked automatically)}
\ExtensionTok{R}
\ExtensionTok{install.packages}\ErrorTok{(}\StringTok{"newpackage"}\KeywordTok{)}
\FunctionTok{quit()}

\CommentTok{\# Work on reports}
\ExtensionTok{vim}\NormalTok{ analysis/report.Rmd       }\CommentTok{\# Edit R Markdown report}
\ExtensionTok{R}
\ExtensionTok{rmarkdown::render}\ErrorTok{(}\StringTok{"analysis/report.Rmd"}\KeywordTok{)}  \CommentTok{\# Generate report}
\FunctionTok{quit()}

\CommentTok{\# Git workflow (from inside container)}
\FunctionTok{git}\NormalTok{ status                    }\CommentTok{\# Check changes}
\FunctionTok{git}\NormalTok{ add .                    }\CommentTok{\# Stage changes}
\FunctionTok{git}\NormalTok{ diff }\AttributeTok{{-}{-}cached}            \CommentTok{\# Review staged changes}
\end{Highlighting}
\end{Shaded}

\subsubsection{3. Exit Container and
Commit}\label{exit-container-and-commit}

\begin{Shaded}
\begin{Highlighting}[]
\CommentTok{\# Exit the development container}
\BuiltInTok{exit}

\CommentTok{\# You\textquotesingle{}re now back on your host system}
\CommentTok{\# Validate dependencies and run final tests}
\FunctionTok{make}\NormalTok{ docker{-}check{-}renv{-}fix    }\CommentTok{\# Auto{-}fix any dependency issues}
\FunctionTok{make}\NormalTok{ docker{-}test             }\CommentTok{\# Run all tests in clean environment}
\FunctionTok{make}\NormalTok{ docker{-}render           }\CommentTok{\# Ensure reports render correctly}

\CommentTok{\# Git workflow {-} commit and push}
\FunctionTok{git}\NormalTok{ status                   }\CommentTok{\# Check what changed}
\FunctionTok{git}\NormalTok{ diff                    }\CommentTok{\# Review changes}

\FunctionTok{git}\NormalTok{ add .}
\FunctionTok{git}\NormalTok{ commit }\AttributeTok{{-}m} \StringTok{"Add data analysis with visualization}

\StringTok{{-} Implement customer segmentation analysis}
\StringTok{{-} Add clustering functions with tests}
\StringTok{{-} Generate summary report with plots}
\StringTok{{-} All tests passing, dependencies validated"}

\FunctionTok{git}\NormalTok{ push origin main        }\CommentTok{\# Push to GitHub}
\end{Highlighting}
\end{Shaded}

\subsubsection{4. What Happens
Automatically}\label{what-happens-automatically}

When you push changes to GitHub:

\begin{enumerate}
\def\labelenumi{\arabic{enumi}.}
\tightlist
\item
  \textbf{GitHub Actions automatically}:

  \begin{itemize}
  \tightlist
  \item
    ✅ Runs R package validation
  \item
    ✅ Executes all tests
  \item
    ✅ Renders analysis reports
  \item
    ✅ Detects if new packages were added
  \end{itemize}
\item
  \textbf{If new packages detected}:

  \begin{itemize}
  \tightlist
  \item
    ✅ Rebuilds Docker image with new packages
  \item
    ✅ Pushes updated image to Docker Hub
  \item
    ✅ Next time you run \texttt{make\ docker-zsh}, you get the updated
    environment
  \end{itemize}
\end{enumerate}

\subsubsection{5. Continue Development
Cycle}\label{continue-development-cycle}

\begin{Shaded}
\begin{Highlighting}[]
\CommentTok{\# Start next iteration}
\FunctionTok{make}\NormalTok{ docker{-}zsh              }\CommentTok{\# Continue with updated environment}
\CommentTok{\# ... more development inside container ...}
\BuiltInTok{exit}
\CommentTok{\# ... commit and push changes ...}
\end{Highlighting}
\end{Shaded}

\subsection{Advanced Development
Patterns}\label{advanced-development-patterns}

\subsubsection{Working with Different
Variants}\label{working-with-different-variants}

\begin{Shaded}
\begin{Highlighting}[]
\CommentTok{\# Switch between different environments for different tasks}
\FunctionTok{make}\NormalTok{ docker{-}zsh              }\CommentTok{\# Use analysis variant for data exploration}
\FunctionTok{make}\NormalTok{ docker{-}rstudio          }\CommentTok{\# Use RStudio for interactive development}
\FunctionTok{make}\NormalTok{ docker{-}verse            }\CommentTok{\# Use publishing variant for report writing}

\CommentTok{\# Each environment has specialized packages for its purpose}
\end{Highlighting}
\end{Shaded}

\subsubsection{Testing and Validation
Workflow}\label{testing-and-validation-workflow}

\begin{Shaded}
\begin{Highlighting}[]
\CommentTok{\# Before committing, validate your work:}
\FunctionTok{make}\NormalTok{ docker{-}test             }\CommentTok{\# Run all automated tests}
\FunctionTok{make}\NormalTok{ docker{-}check            }\CommentTok{\# R CMD check validation}
\FunctionTok{make}\NormalTok{ docker{-}render           }\CommentTok{\# Ensure all reports render}

\CommentTok{\# Fix any issues before committing}
\FunctionTok{make}\NormalTok{ docker{-}zsh}
\CommentTok{\# ... fix issues inside container ...}
\BuiltInTok{exit}
\end{Highlighting}
\end{Shaded}

\subsubsection{Package Development
Focus}\label{package-development-focus}

\begin{Shaded}
\begin{Highlighting}[]
\CommentTok{\# Inside container {-} R package development workflow}
\ExtensionTok{R}
\FunctionTok{devtools::check()}            \CommentTok{\# Full package check}
\FunctionTok{devtools::build()}            \CommentTok{\# Build package}
\FunctionTok{devtools::install()}          \CommentTok{\# Install your package}
\ExtensionTok{usethis::use\_test}\ErrorTok{(}\StringTok{"myfunction"}\KeywordTok{)}  \CommentTok{\# Create test file}
\FunctionTok{quit()}

\CommentTok{\# Document and check}
\FunctionTok{make}\NormalTok{ docker{-}document         }\CommentTok{\# Generate documentation}
\FunctionTok{make}\NormalTok{ docker{-}check           }\CommentTok{\# Full package validation}
\end{Highlighting}
\end{Shaded}

\subsection{Practical Example: Penguin Bill
Analysis}\label{practical-example-penguin-bill-analysis}

Let's walk through a complete example using the Palmer penguins dataset
to demonstrate the iterative development workflow.

\subsubsection{Step 1: Create Project and Initial
Analysis}\label{step-1-create-project-and-initial-analysis}

\begin{Shaded}
\begin{Highlighting}[]
\CommentTok{\# Set up the project}
\ExtensionTok{zzcollab} \AttributeTok{{-}i} \AttributeTok{{-}p}\NormalTok{ penguin{-}analysis }\AttributeTok{{-}{-}github}

\CommentTok{\# Start development environment}
\BuiltInTok{cd}\NormalTok{ penguin{-}analysis}
\FunctionTok{make}\NormalTok{ docker{-}zsh}
\end{Highlighting}
\end{Shaded}

\textbf{Inside the container - Create initial analysis script:}

\begin{Shaded}
\begin{Highlighting}[]
\CommentTok{\# Create the analysis script}
\ExtensionTok{vim}\NormalTok{ scripts/01\_penguin\_exploration.R}
\end{Highlighting}
\end{Shaded}

\textbf{Contents of \texttt{scripts/01\_penguin\_exploration.R}:}

\begin{Shaded}
\begin{Highlighting}[]
\CommentTok{\#\textquotesingle{} Penguin Bill Analysis}
\CommentTok{\#\textquotesingle{} Explore relationship between bill depth and log of bill length}

\CommentTok{\# Load required packages}
\FunctionTok{library}\NormalTok{(palmerpenguins)}
\FunctionTok{library}\NormalTok{(ggplot2)}
\FunctionTok{library}\NormalTok{(dplyr)}

\CommentTok{\#\textquotesingle{} Create scatter plot of bill depth vs log(bill length)}
\CommentTok{\#\textquotesingle{} @return ggplot object}
\NormalTok{create\_bill\_plot }\OtherTok{\textless{}{-}} \ControlFlowTok{function}\NormalTok{() \{}
\NormalTok{  penguins }\SpecialCharTok{\%\textgreater{}\%}
    \FunctionTok{filter}\NormalTok{(}\SpecialCharTok{!}\FunctionTok{is.na}\NormalTok{(bill\_length\_mm), }\SpecialCharTok{!}\FunctionTok{is.na}\NormalTok{(bill\_depth\_mm)) }\SpecialCharTok{\%\textgreater{}\%}
    \FunctionTok{ggplot}\NormalTok{(}\FunctionTok{aes}\NormalTok{(}\AttributeTok{x =} \FunctionTok{log}\NormalTok{(bill\_length\_mm), }\AttributeTok{y =}\NormalTok{ bill\_depth\_mm)) }\SpecialCharTok{+}
    \FunctionTok{geom\_point}\NormalTok{(}\FunctionTok{aes}\NormalTok{(}\AttributeTok{color =}\NormalTok{ species), }\AttributeTok{alpha =} \FloatTok{0.7}\NormalTok{, }\AttributeTok{size =} \DecValTok{2}\NormalTok{) }\SpecialCharTok{+}
    \FunctionTok{labs}\NormalTok{(}
      \AttributeTok{title =} \StringTok{"Penguin Bill Depth vs Log(Bill Length)"}\NormalTok{,}
      \AttributeTok{x =} \StringTok{"Log(Bill Length) (mm)"}\NormalTok{,}
      \AttributeTok{y =} \StringTok{"Bill Depth (mm)"}\NormalTok{,}
      \AttributeTok{color =} \StringTok{"Species"}
\NormalTok{    ) }\SpecialCharTok{+}
    \FunctionTok{theme\_minimal}\NormalTok{() }\SpecialCharTok{+}
    \FunctionTok{theme}\NormalTok{(}\AttributeTok{legend.position =} \StringTok{"bottom"}\NormalTok{)}
\NormalTok{\}}

\CommentTok{\# Create and display the plot}
\NormalTok{bill\_plot }\OtherTok{\textless{}{-}} \FunctionTok{create\_bill\_plot}\NormalTok{()}
\FunctionTok{print}\NormalTok{(bill\_plot)}

\CommentTok{\# Save the plot}
\FunctionTok{ggsave}\NormalTok{(}\StringTok{"figures/bill\_analysis.png"}\NormalTok{, bill\_plot, }\AttributeTok{width =} \DecValTok{8}\NormalTok{, }\AttributeTok{height =} \DecValTok{6}\NormalTok{, }\AttributeTok{dpi =} \DecValTok{300}\NormalTok{)}

\FunctionTok{cat}\NormalTok{(}\StringTok{"Analysis complete! Plot saved to figures/bill\_analysis.png}\SpecialCharTok{\textbackslash{}n}\StringTok{"}\NormalTok{)}
\end{Highlighting}
\end{Shaded}

\textbf{Create the function file:}

\begin{Shaded}
\begin{Highlighting}[]
\CommentTok{\# Create R function file}
\ExtensionTok{vim}\NormalTok{ R/penguin\_functions.R}
\end{Highlighting}
\end{Shaded}

\textbf{Contents of \texttt{R/penguin\_functions.R}:}

\begin{Shaded}
\begin{Highlighting}[]
\CommentTok{\#\textquotesingle{} Create scatter plot of bill depth vs log(bill length)}
\CommentTok{\#\textquotesingle{} }
\CommentTok{\#\textquotesingle{} @param data Data frame containing penguin data (default: palmerpenguins::penguins)}
\CommentTok{\#\textquotesingle{} @return ggplot object}
\CommentTok{\#\textquotesingle{} @export}
\CommentTok{\#\textquotesingle{} @examples}
\CommentTok{\#\textquotesingle{} plot \textless{}{-} create\_bill\_plot()}
\CommentTok{\#\textquotesingle{} print(plot)}
\NormalTok{create\_bill\_plot }\OtherTok{\textless{}{-}} \ControlFlowTok{function}\NormalTok{(}\AttributeTok{data =}\NormalTok{ palmerpenguins}\SpecialCharTok{::}\NormalTok{penguins) \{}
  \ControlFlowTok{if}\NormalTok{ (}\SpecialCharTok{!}\FunctionTok{requireNamespace}\NormalTok{(}\StringTok{"ggplot2"}\NormalTok{, }\AttributeTok{quietly =} \ConstantTok{TRUE}\NormalTok{)) \{}
    \FunctionTok{stop}\NormalTok{(}\StringTok{"ggplot2 package is required"}\NormalTok{)}
\NormalTok{  \}}
  \ControlFlowTok{if}\NormalTok{ (}\SpecialCharTok{!}\FunctionTok{requireNamespace}\NormalTok{(}\StringTok{"dplyr"}\NormalTok{, }\AttributeTok{quietly =} \ConstantTok{TRUE}\NormalTok{)) \{}
    \FunctionTok{stop}\NormalTok{(}\StringTok{"dplyr package is required"}\NormalTok{)}
\NormalTok{  \}}
  
\NormalTok{  data }\SpecialCharTok{\%\textgreater{}\%}
\NormalTok{    dplyr}\SpecialCharTok{::}\FunctionTok{filter}\NormalTok{(}\SpecialCharTok{!}\FunctionTok{is.na}\NormalTok{(bill\_length\_mm), }\SpecialCharTok{!}\FunctionTok{is.na}\NormalTok{(bill\_depth\_mm)) }\SpecialCharTok{\%\textgreater{}\%}
\NormalTok{    ggplot2}\SpecialCharTok{::}\FunctionTok{ggplot}\NormalTok{(ggplot2}\SpecialCharTok{::}\FunctionTok{aes}\NormalTok{(}\AttributeTok{x =} \FunctionTok{log}\NormalTok{(bill\_length\_mm), }\AttributeTok{y =}\NormalTok{ bill\_depth\_mm)) }\SpecialCharTok{+}
\NormalTok{    ggplot2}\SpecialCharTok{::}\FunctionTok{geom\_point}\NormalTok{(ggplot2}\SpecialCharTok{::}\FunctionTok{aes}\NormalTok{(}\AttributeTok{color =}\NormalTok{ species), }\AttributeTok{alpha =} \FloatTok{0.7}\NormalTok{, }\AttributeTok{size =} \DecValTok{2}\NormalTok{) }\SpecialCharTok{+}
\NormalTok{    ggplot2}\SpecialCharTok{::}\FunctionTok{labs}\NormalTok{(}
      \AttributeTok{title =} \StringTok{"Penguin Bill Depth vs Log(Bill Length)"}\NormalTok{,}
      \AttributeTok{x =} \StringTok{"Log(Bill Length) (mm)"}\NormalTok{,}
      \AttributeTok{y =} \StringTok{"Bill Depth (mm)"}\NormalTok{,}
      \AttributeTok{color =} \StringTok{"Species"}
\NormalTok{    ) }\SpecialCharTok{+}
\NormalTok{    ggplot2}\SpecialCharTok{::}\FunctionTok{theme\_minimal}\NormalTok{() }\SpecialCharTok{+}
\NormalTok{    ggplot2}\SpecialCharTok{::}\FunctionTok{theme}\NormalTok{(}\AttributeTok{legend.position =} \StringTok{"bottom"}\NormalTok{)}
\NormalTok{\}}
\end{Highlighting}
\end{Shaded}

\textbf{Create tests for the function:}

\begin{Shaded}
\begin{Highlighting}[]
\CommentTok{\# Create test file}
\ExtensionTok{vim}\NormalTok{ tests/testthat/test{-}penguin\_functions.R}
\end{Highlighting}
\end{Shaded}

\textbf{Contents of \texttt{tests/testthat/test-penguin\_functions.R}:}

\begin{Shaded}
\begin{Highlighting}[]
\FunctionTok{test\_that}\NormalTok{(}\StringTok{"create\_bill\_plot works correctly"}\NormalTok{, \{}
  \CommentTok{\# Test with default data}
\NormalTok{  plot }\OtherTok{\textless{}{-}} \FunctionTok{create\_bill\_plot}\NormalTok{()}
  
  \CommentTok{\# Check that it returns a ggplot object}
  \FunctionTok{expect\_s3\_class}\NormalTok{(plot, }\StringTok{"ggplot"}\NormalTok{)}
  
  \CommentTok{\# Check plot components}
  \FunctionTok{expect\_equal}\NormalTok{(plot}\SpecialCharTok{$}\NormalTok{labels}\SpecialCharTok{$}\NormalTok{title, }\StringTok{"Penguin Bill Depth vs Log(Bill Length)"}\NormalTok{)}
  \FunctionTok{expect\_equal}\NormalTok{(plot}\SpecialCharTok{$}\NormalTok{labels}\SpecialCharTok{$}\NormalTok{x, }\StringTok{"Log(Bill Length) (mm)"}\NormalTok{)}
  \FunctionTok{expect\_equal}\NormalTok{(plot}\SpecialCharTok{$}\NormalTok{labels}\SpecialCharTok{$}\NormalTok{y, }\StringTok{"Bill Depth (mm)"}\NormalTok{)}
  \FunctionTok{expect\_equal}\NormalTok{(plot}\SpecialCharTok{$}\NormalTok{labels}\SpecialCharTok{$}\NormalTok{colour, }\StringTok{"Species"}\NormalTok{)}
\NormalTok{\})}

\FunctionTok{test\_that}\NormalTok{(}\StringTok{"create\_bill\_plot handles custom data"}\NormalTok{, \{}
  \CommentTok{\# Create test data}
\NormalTok{  test\_data }\OtherTok{\textless{}{-}} \FunctionTok{data.frame}\NormalTok{(}
    \AttributeTok{bill\_length\_mm =} \FunctionTok{c}\NormalTok{(}\DecValTok{40}\NormalTok{, }\DecValTok{45}\NormalTok{, }\DecValTok{50}\NormalTok{),}
    \AttributeTok{bill\_depth\_mm =} \FunctionTok{c}\NormalTok{(}\DecValTok{18}\NormalTok{, }\DecValTok{20}\NormalTok{, }\DecValTok{22}\NormalTok{),}
    \AttributeTok{species =} \FunctionTok{c}\NormalTok{(}\StringTok{"A"}\NormalTok{, }\StringTok{"B"}\NormalTok{, }\StringTok{"C"}\NormalTok{)}
\NormalTok{  )}
  
\NormalTok{  plot }\OtherTok{\textless{}{-}} \FunctionTok{create\_bill\_plot}\NormalTok{(test\_data)}
  \FunctionTok{expect\_s3\_class}\NormalTok{(plot, }\StringTok{"ggplot"}\NormalTok{)}
\NormalTok{\})}

\FunctionTok{test\_that}\NormalTok{(}\StringTok{"create\_bill\_plot handles missing values"}\NormalTok{, \{}
  \CommentTok{\# Create test data with NA values}
\NormalTok{  test\_data }\OtherTok{\textless{}{-}} \FunctionTok{data.frame}\NormalTok{(}
    \AttributeTok{bill\_length\_mm =} \FunctionTok{c}\NormalTok{(}\DecValTok{40}\NormalTok{, }\ConstantTok{NA}\NormalTok{, }\DecValTok{50}\NormalTok{),}
    \AttributeTok{bill\_depth\_mm =} \FunctionTok{c}\NormalTok{(}\DecValTok{18}\NormalTok{, }\DecValTok{20}\NormalTok{, }\ConstantTok{NA}\NormalTok{),}
    \AttributeTok{species =} \FunctionTok{c}\NormalTok{(}\StringTok{"A"}\NormalTok{, }\StringTok{"B"}\NormalTok{, }\StringTok{"C"}\NormalTok{)}
\NormalTok{  )}
  
\NormalTok{  plot }\OtherTok{\textless{}{-}} \FunctionTok{create\_bill\_plot}\NormalTok{(test\_data)}
  \FunctionTok{expect\_s3\_class}\NormalTok{(plot, }\StringTok{"ggplot"}\NormalTok{)}
  
  \CommentTok{\# Should have only 1 point after filtering NAs}
  \FunctionTok{expect\_equal}\NormalTok{(}\FunctionTok{nrow}\NormalTok{(plot}\SpecialCharTok{$}\NormalTok{data), }\DecValTok{1}\NormalTok{)}
\NormalTok{\})}
\end{Highlighting}
\end{Shaded}

\textbf{Test and run the analysis:}

\begin{Shaded}
\begin{Highlighting}[]
\CommentTok{\# Install required packages}
\ExtensionTok{R}
\ExtensionTok{install.packages}\ErrorTok{(}\ExtensionTok{c}\ErrorTok{(}\StringTok{"palmerpenguins"}\ExtensionTok{,} \StringTok{"ggplot2"}\NormalTok{, }\StringTok{"dplyr"}\KeywordTok{))}
\FunctionTok{quit()}

\CommentTok{\# Test the function}
\ExtensionTok{R}
\FunctionTok{devtools::load\_all()}
\FunctionTok{devtools::test()}
\FunctionTok{quit()}

\CommentTok{\# Run the analysis script}
\FunctionTok{mkdir} \AttributeTok{{-}p}\NormalTok{ figures}
\ExtensionTok{R} \AttributeTok{{-}{-}vanilla} \OperatorTok{\textless{}}\NormalTok{ scripts/01\_penguin\_exploration.R}

\CommentTok{\# Check the git status}
\FunctionTok{git}\NormalTok{ status}
\FunctionTok{git}\NormalTok{ add .}
\FunctionTok{git}\NormalTok{ diff }\AttributeTok{{-}{-}cached}
\end{Highlighting}
\end{Shaded}

\subsubsection{Step 2: Exit Container and First
Commit}\label{step-2-exit-container-and-first-commit}

\begin{Shaded}
\begin{Highlighting}[]
\CommentTok{\# Exit the development container}
\BuiltInTok{exit}

\CommentTok{\# Validate dependencies and test}
\FunctionTok{make}\NormalTok{ docker{-}check{-}renv{-}fix    }\CommentTok{\# Auto{-}add new packages to renv.lock}
\FunctionTok{make}\NormalTok{ docker{-}test             }\CommentTok{\# Run tests in clean environment}

\CommentTok{\# First commit and push}
\FunctionTok{git}\NormalTok{ add .}
\FunctionTok{git}\NormalTok{ commit }\AttributeTok{{-}m} \StringTok{"Add initial penguin bill analysis}

\StringTok{{-} Create scatter plot of bill depth vs log(bill length)}
\StringTok{{-} Add create\_bill\_plot() function with comprehensive tests}
\StringTok{{-} Generate publication{-}quality figure}
\StringTok{{-} All tests passing, dependencies tracked"}

\FunctionTok{git}\NormalTok{ push origin main}
\end{Highlighting}
\end{Shaded}

\subsubsection{Step 3: Continue Analysis - Add Regression
Line}\label{step-3-continue-analysis---add-regression-line}

After the first push, continue with enhanced analysis:

\begin{Shaded}
\begin{Highlighting}[]
\CommentTok{\# Start development environment again}
\FunctionTok{make}\NormalTok{ docker{-}zsh}

\CommentTok{\# Update the analysis script}
\ExtensionTok{vim}\NormalTok{ scripts/01\_penguin\_exploration.R}
\end{Highlighting}
\end{Shaded}

\textbf{Updated \texttt{scripts/01\_penguin\_exploration.R}:}

\begin{Shaded}
\begin{Highlighting}[]
\CommentTok{\#\textquotesingle{} Penguin Bill Analysis {-} Enhanced with Regression}
\CommentTok{\#\textquotesingle{} Explore relationship between bill depth and log of bill length}

\CommentTok{\# Load required packages}
\FunctionTok{library}\NormalTok{(palmerpenguins)}
\FunctionTok{library}\NormalTok{(ggplot2)}
\FunctionTok{library}\NormalTok{(dplyr)}
\FunctionTok{library}\NormalTok{(broom)}

\CommentTok{\#\textquotesingle{} Create enhanced scatter plot with regression line}
\CommentTok{\#\textquotesingle{} @return ggplot object}
\NormalTok{create\_enhanced\_bill\_plot }\OtherTok{\textless{}{-}} \ControlFlowTok{function}\NormalTok{() \{}
\NormalTok{  penguins }\SpecialCharTok{\%\textgreater{}\%}
    \FunctionTok{filter}\NormalTok{(}\SpecialCharTok{!}\FunctionTok{is.na}\NormalTok{(bill\_length\_mm), }\SpecialCharTok{!}\FunctionTok{is.na}\NormalTok{(bill\_depth\_mm)) }\SpecialCharTok{\%\textgreater{}\%}
    \FunctionTok{ggplot}\NormalTok{(}\FunctionTok{aes}\NormalTok{(}\AttributeTok{x =} \FunctionTok{log}\NormalTok{(bill\_length\_mm), }\AttributeTok{y =}\NormalTok{ bill\_depth\_mm)) }\SpecialCharTok{+}
    \FunctionTok{geom\_point}\NormalTok{(}\FunctionTok{aes}\NormalTok{(}\AttributeTok{color =}\NormalTok{ species), }\AttributeTok{alpha =} \FloatTok{0.7}\NormalTok{, }\AttributeTok{size =} \DecValTok{2}\NormalTok{) }\SpecialCharTok{+}
    \FunctionTok{geom\_smooth}\NormalTok{(}\AttributeTok{method =} \StringTok{"lm"}\NormalTok{, }\AttributeTok{se =} \ConstantTok{TRUE}\NormalTok{, }\AttributeTok{color =} \StringTok{"black"}\NormalTok{, }\AttributeTok{linetype =} \StringTok{"dashed"}\NormalTok{) }\SpecialCharTok{+}
    \FunctionTok{labs}\NormalTok{(}
      \AttributeTok{title =} \StringTok{"Penguin Bill Depth vs Log(Bill Length) with Regression Line"}\NormalTok{,}
      \AttributeTok{x =} \StringTok{"Log(Bill Length) (mm)"}\NormalTok{,}
      \AttributeTok{y =} \StringTok{"Bill Depth (mm)"}\NormalTok{,}
      \AttributeTok{color =} \StringTok{"Species"}\NormalTok{,}
      \AttributeTok{caption =} \StringTok{"Dashed line shows linear regression fit with 95\% confidence interval"}
\NormalTok{    ) }\SpecialCharTok{+}
    \FunctionTok{theme\_minimal}\NormalTok{() }\SpecialCharTok{+}
    \FunctionTok{theme}\NormalTok{(}\AttributeTok{legend.position =} \StringTok{"bottom"}\NormalTok{)}
\NormalTok{\}}

\CommentTok{\#\textquotesingle{} Fit linear model for bill depth vs log(bill length)}
\CommentTok{\#\textquotesingle{} @return list with model object and summary statistics}
\NormalTok{fit\_bill\_model }\OtherTok{\textless{}{-}} \ControlFlowTok{function}\NormalTok{() \{}
\NormalTok{  clean\_data }\OtherTok{\textless{}{-}}\NormalTok{ penguins }\SpecialCharTok{\%\textgreater{}\%}
    \FunctionTok{filter}\NormalTok{(}\SpecialCharTok{!}\FunctionTok{is.na}\NormalTok{(bill\_length\_mm), }\SpecialCharTok{!}\FunctionTok{is.na}\NormalTok{(bill\_depth\_mm)) }\SpecialCharTok{\%\textgreater{}\%}
    \FunctionTok{mutate}\NormalTok{(}\AttributeTok{log\_bill\_length =} \FunctionTok{log}\NormalTok{(bill\_length\_mm))}
  
\NormalTok{  model }\OtherTok{\textless{}{-}} \FunctionTok{lm}\NormalTok{(bill\_depth\_mm }\SpecialCharTok{\textasciitilde{}}\NormalTok{ log\_bill\_length, }\AttributeTok{data =}\NormalTok{ clean\_data)}
  
  \FunctionTok{list}\NormalTok{(}
    \AttributeTok{model =}\NormalTok{ model,}
    \AttributeTok{summary =} \FunctionTok{summary}\NormalTok{(model),}
    \AttributeTok{r\_squared =} \FunctionTok{summary}\NormalTok{(model)}\SpecialCharTok{$}\NormalTok{r.squared,}
    \AttributeTok{coefficients =} \FunctionTok{tidy}\NormalTok{(model)}
\NormalTok{  )}
\NormalTok{\}}

\CommentTok{\# Create enhanced plot}
\NormalTok{enhanced\_plot }\OtherTok{\textless{}{-}} \FunctionTok{create\_enhanced\_bill\_plot}\NormalTok{()}
\FunctionTok{print}\NormalTok{(enhanced\_plot)}

\CommentTok{\# Fit regression model}
\NormalTok{model\_results }\OtherTok{\textless{}{-}} \FunctionTok{fit\_bill\_model}\NormalTok{()}
\FunctionTok{cat}\NormalTok{(}\StringTok{"}\SpecialCharTok{\textbackslash{}n}\StringTok{Regression Results:}\SpecialCharTok{\textbackslash{}n}\StringTok{"}\NormalTok{)}
\FunctionTok{cat}\NormalTok{(}\StringTok{"R{-}squared:"}\NormalTok{, }\FunctionTok{round}\NormalTok{(model\_results}\SpecialCharTok{$}\NormalTok{r\_squared, }\DecValTok{3}\NormalTok{), }\StringTok{"}\SpecialCharTok{\textbackslash{}n}\StringTok{"}\NormalTok{)}
\FunctionTok{print}\NormalTok{(model\_results}\SpecialCharTok{$}\NormalTok{coefficients)}

\CommentTok{\# Save outputs}
\FunctionTok{ggsave}\NormalTok{(}\StringTok{"figures/bill\_analysis\_with\_regression.png"}\NormalTok{, enhanced\_plot, }
       \AttributeTok{width =} \DecValTok{8}\NormalTok{, }\AttributeTok{height =} \DecValTok{6}\NormalTok{, }\AttributeTok{dpi =} \DecValTok{300}\NormalTok{)}

\CommentTok{\# Save model results}
\FunctionTok{saveRDS}\NormalTok{(model\_results, }\StringTok{"results/bill\_model.rds"}\NormalTok{)}

\FunctionTok{cat}\NormalTok{(}\StringTok{"}\SpecialCharTok{\textbackslash{}n}\StringTok{Enhanced analysis complete!}\SpecialCharTok{\textbackslash{}n}\StringTok{"}\NormalTok{)}
\FunctionTok{cat}\NormalTok{(}\StringTok{"Plot: figures/bill\_analysis\_with\_regression.png}\SpecialCharTok{\textbackslash{}n}\StringTok{"}\NormalTok{)}
\FunctionTok{cat}\NormalTok{(}\StringTok{"Model: results/bill\_model.rds}\SpecialCharTok{\textbackslash{}n}\StringTok{"}\NormalTok{)}
\end{Highlighting}
\end{Shaded}

\textbf{Update the function file:}

\begin{Shaded}
\begin{Highlighting}[]
\ExtensionTok{vim}\NormalTok{ R/penguin\_functions.R}
\end{Highlighting}
\end{Shaded}

\textbf{Add to \texttt{R/penguin\_functions.R}:}

\begin{Shaded}
\begin{Highlighting}[]
\CommentTok{\#\textquotesingle{} Create enhanced scatter plot with regression line}
\CommentTok{\#\textquotesingle{} }
\CommentTok{\#\textquotesingle{} @param data Data frame containing penguin data (default: palmerpenguins::penguins)}
\CommentTok{\#\textquotesingle{} @return ggplot object}
\CommentTok{\#\textquotesingle{} @export}
\NormalTok{create\_enhanced\_bill\_plot }\OtherTok{\textless{}{-}} \ControlFlowTok{function}\NormalTok{(}\AttributeTok{data =}\NormalTok{ palmerpenguins}\SpecialCharTok{::}\NormalTok{penguins) \{}
  \ControlFlowTok{if}\NormalTok{ (}\SpecialCharTok{!}\FunctionTok{requireNamespace}\NormalTok{(}\StringTok{"ggplot2"}\NormalTok{, }\AttributeTok{quietly =} \ConstantTok{TRUE}\NormalTok{)) \{}
    \FunctionTok{stop}\NormalTok{(}\StringTok{"ggplot2 package is required"}\NormalTok{)}
\NormalTok{  \}}
  \ControlFlowTok{if}\NormalTok{ (}\SpecialCharTok{!}\FunctionTok{requireNamespace}\NormalTok{(}\StringTok{"dplyr"}\NormalTok{, }\AttributeTok{quietly =} \ConstantTok{TRUE}\NormalTok{)) \{}
    \FunctionTok{stop}\NormalTok{(}\StringTok{"dplyr package is required"}\NormalTok{)}
\NormalTok{  \}}
  
\NormalTok{  data }\SpecialCharTok{\%\textgreater{}\%}
\NormalTok{    dplyr}\SpecialCharTok{::}\FunctionTok{filter}\NormalTok{(}\SpecialCharTok{!}\FunctionTok{is.na}\NormalTok{(bill\_length\_mm), }\SpecialCharTok{!}\FunctionTok{is.na}\NormalTok{(bill\_depth\_mm)) }\SpecialCharTok{\%\textgreater{}\%}
\NormalTok{    ggplot2}\SpecialCharTok{::}\FunctionTok{ggplot}\NormalTok{(ggplot2}\SpecialCharTok{::}\FunctionTok{aes}\NormalTok{(}\AttributeTok{x =} \FunctionTok{log}\NormalTok{(bill\_length\_mm), }\AttributeTok{y =}\NormalTok{ bill\_depth\_mm)) }\SpecialCharTok{+}
\NormalTok{    ggplot2}\SpecialCharTok{::}\FunctionTok{geom\_point}\NormalTok{(ggplot2}\SpecialCharTok{::}\FunctionTok{aes}\NormalTok{(}\AttributeTok{color =}\NormalTok{ species), }\AttributeTok{alpha =} \FloatTok{0.7}\NormalTok{, }\AttributeTok{size =} \DecValTok{2}\NormalTok{) }\SpecialCharTok{+}
\NormalTok{    ggplot2}\SpecialCharTok{::}\FunctionTok{geom\_smooth}\NormalTok{(}\AttributeTok{method =} \StringTok{"lm"}\NormalTok{, }\AttributeTok{se =} \ConstantTok{TRUE}\NormalTok{, }\AttributeTok{color =} \StringTok{"black"}\NormalTok{, }\AttributeTok{linetype =} \StringTok{"dashed"}\NormalTok{) }\SpecialCharTok{+}
\NormalTok{    ggplot2}\SpecialCharTok{::}\FunctionTok{labs}\NormalTok{(}
      \AttributeTok{title =} \StringTok{"Penguin Bill Depth vs Log(Bill Length) with Regression Line"}\NormalTok{,}
      \AttributeTok{x =} \StringTok{"Log(Bill Length) (mm)"}\NormalTok{,}
      \AttributeTok{y =} \StringTok{"Bill Depth (mm)"}\NormalTok{,}
      \AttributeTok{color =} \StringTok{"Species"}\NormalTok{,}
      \AttributeTok{caption =} \StringTok{"Dashed line shows linear regression fit with 95\% confidence interval"}
\NormalTok{    ) }\SpecialCharTok{+}
\NormalTok{    ggplot2}\SpecialCharTok{::}\FunctionTok{theme\_minimal}\NormalTok{() }\SpecialCharTok{+}
\NormalTok{    ggplot2}\SpecialCharTok{::}\FunctionTok{theme}\NormalTok{(}\AttributeTok{legend.position =} \StringTok{"bottom"}\NormalTok{)}
\NormalTok{\}}

\CommentTok{\#\textquotesingle{} Fit linear model for bill depth vs log(bill length)}
\CommentTok{\#\textquotesingle{} }
\CommentTok{\#\textquotesingle{} @param data Data frame containing penguin data (default: palmerpenguins::penguins)}
\CommentTok{\#\textquotesingle{} @return list with model object and summary statistics}
\CommentTok{\#\textquotesingle{} @export}
\NormalTok{fit\_bill\_model }\OtherTok{\textless{}{-}} \ControlFlowTok{function}\NormalTok{(}\AttributeTok{data =}\NormalTok{ palmerpenguins}\SpecialCharTok{::}\NormalTok{penguins) \{}
  \ControlFlowTok{if}\NormalTok{ (}\SpecialCharTok{!}\FunctionTok{requireNamespace}\NormalTok{(}\StringTok{"dplyr"}\NormalTok{, }\AttributeTok{quietly =} \ConstantTok{TRUE}\NormalTok{)) \{}
    \FunctionTok{stop}\NormalTok{(}\StringTok{"dplyr package is required"}\NormalTok{)}
\NormalTok{  \}}
  \ControlFlowTok{if}\NormalTok{ (}\SpecialCharTok{!}\FunctionTok{requireNamespace}\NormalTok{(}\StringTok{"broom"}\NormalTok{, }\AttributeTok{quietly =} \ConstantTok{TRUE}\NormalTok{)) \{}
    \FunctionTok{stop}\NormalTok{(}\StringTok{"broom package is required"}\NormalTok{)}
\NormalTok{  \}}
  
\NormalTok{  clean\_data }\OtherTok{\textless{}{-}}\NormalTok{ data }\SpecialCharTok{\%\textgreater{}\%}
\NormalTok{    dplyr}\SpecialCharTok{::}\FunctionTok{filter}\NormalTok{(}\SpecialCharTok{!}\FunctionTok{is.na}\NormalTok{(bill\_length\_mm), }\SpecialCharTok{!}\FunctionTok{is.na}\NormalTok{(bill\_depth\_mm)) }\SpecialCharTok{\%\textgreater{}\%}
\NormalTok{    dplyr}\SpecialCharTok{::}\FunctionTok{mutate}\NormalTok{(}\AttributeTok{log\_bill\_length =} \FunctionTok{log}\NormalTok{(bill\_length\_mm))}
  
\NormalTok{  model }\OtherTok{\textless{}{-}} \FunctionTok{lm}\NormalTok{(bill\_depth\_mm }\SpecialCharTok{\textasciitilde{}}\NormalTok{ log\_bill\_length, }\AttributeTok{data =}\NormalTok{ clean\_data)}
  
  \FunctionTok{list}\NormalTok{(}
    \AttributeTok{model =}\NormalTok{ model,}
    \AttributeTok{summary =} \FunctionTok{summary}\NormalTok{(model),}
    \AttributeTok{r\_squared =} \FunctionTok{summary}\NormalTok{(model)}\SpecialCharTok{$}\NormalTok{r.squared,}
    \AttributeTok{coefficients =}\NormalTok{ broom}\SpecialCharTok{::}\FunctionTok{tidy}\NormalTok{(model)}
\NormalTok{  )}
\NormalTok{\}}
\end{Highlighting}
\end{Shaded}

\textbf{Add tests for new functions:}

\begin{Shaded}
\begin{Highlighting}[]
\ExtensionTok{vim}\NormalTok{ tests/testthat/test{-}penguin\_functions.R}
\end{Highlighting}
\end{Shaded}

\textbf{Add to test file:}

\begin{Shaded}
\begin{Highlighting}[]
\FunctionTok{test\_that}\NormalTok{(}\StringTok{"create\_enhanced\_bill\_plot works correctly"}\NormalTok{, \{}
\NormalTok{  plot }\OtherTok{\textless{}{-}} \FunctionTok{create\_enhanced\_bill\_plot}\NormalTok{()}
  
  \FunctionTok{expect\_s3\_class}\NormalTok{(plot, }\StringTok{"ggplot"}\NormalTok{)}
  \FunctionTok{expect\_equal}\NormalTok{(plot}\SpecialCharTok{$}\NormalTok{labels}\SpecialCharTok{$}\NormalTok{title, }
               \StringTok{"Penguin Bill Depth vs Log(Bill Length) with Regression Line"}\NormalTok{)}
  \FunctionTok{expect\_true}\NormalTok{(}\FunctionTok{grepl}\NormalTok{(}\StringTok{"regression"}\NormalTok{, plot}\SpecialCharTok{$}\NormalTok{labels}\SpecialCharTok{$}\NormalTok{caption, }\AttributeTok{ignore.case =} \ConstantTok{TRUE}\NormalTok{))}
\NormalTok{\})}

\FunctionTok{test\_that}\NormalTok{(}\StringTok{"fit\_bill\_model returns correct structure"}\NormalTok{, \{}
\NormalTok{  model\_results }\OtherTok{\textless{}{-}} \FunctionTok{fit\_bill\_model}\NormalTok{()}
  
  \FunctionTok{expect\_type}\NormalTok{(model\_results, }\StringTok{"list"}\NormalTok{)}
  \FunctionTok{expect\_true}\NormalTok{(}\StringTok{"model"} \SpecialCharTok{\%in\%} \FunctionTok{names}\NormalTok{(model\_results))}
  \FunctionTok{expect\_true}\NormalTok{(}\StringTok{"summary"} \SpecialCharTok{\%in\%} \FunctionTok{names}\NormalTok{(model\_results))}
  \FunctionTok{expect\_true}\NormalTok{(}\StringTok{"r\_squared"} \SpecialCharTok{\%in\%} \FunctionTok{names}\NormalTok{(model\_results))}
  \FunctionTok{expect\_true}\NormalTok{(}\StringTok{"coefficients"} \SpecialCharTok{\%in\%} \FunctionTok{names}\NormalTok{(model\_results))}
  
  \FunctionTok{expect\_s3\_class}\NormalTok{(model\_results}\SpecialCharTok{$}\NormalTok{model, }\StringTok{"lm"}\NormalTok{)}
  \FunctionTok{expect\_type}\NormalTok{(model\_results}\SpecialCharTok{$}\NormalTok{r\_squared, }\StringTok{"double"}\NormalTok{)}
  \FunctionTok{expect\_s3\_class}\NormalTok{(model\_results}\SpecialCharTok{$}\NormalTok{coefficients, }\StringTok{"data.frame"}\NormalTok{)}
\NormalTok{\})}

\FunctionTok{test\_that}\NormalTok{(}\StringTok{"fit\_bill\_model handles custom data"}\NormalTok{, \{}
\NormalTok{  test\_data }\OtherTok{\textless{}{-}} \FunctionTok{data.frame}\NormalTok{(}
    \AttributeTok{bill\_length\_mm =} \FunctionTok{c}\NormalTok{(}\DecValTok{40}\NormalTok{, }\DecValTok{45}\NormalTok{, }\DecValTok{50}\NormalTok{, }\DecValTok{55}\NormalTok{),}
    \AttributeTok{bill\_depth\_mm =} \FunctionTok{c}\NormalTok{(}\DecValTok{18}\NormalTok{, }\DecValTok{19}\NormalTok{, }\DecValTok{20}\NormalTok{, }\DecValTok{21}\NormalTok{),}
    \AttributeTok{species =} \FunctionTok{c}\NormalTok{(}\StringTok{"A"}\NormalTok{, }\StringTok{"B"}\NormalTok{, }\StringTok{"C"}\NormalTok{, }\StringTok{"A"}\NormalTok{)}
\NormalTok{  )}
  
\NormalTok{  model\_results }\OtherTok{\textless{}{-}} \FunctionTok{fit\_bill\_model}\NormalTok{(test\_data)}
  \FunctionTok{expect\_s3\_class}\NormalTok{(model\_results}\SpecialCharTok{$}\NormalTok{model, }\StringTok{"lm"}\NormalTok{)}
  \FunctionTok{expect\_true}\NormalTok{(model\_results}\SpecialCharTok{$}\NormalTok{r\_squared }\SpecialCharTok{\textgreater{}=} \DecValTok{0} \SpecialCharTok{\&\&}\NormalTok{ model\_results}\SpecialCharTok{$}\NormalTok{r\_squared }\SpecialCharTok{\textless{}=} \DecValTok{1}\NormalTok{)}
\NormalTok{\})}
\end{Highlighting}
\end{Shaded}

\textbf{Test and run the enhanced analysis:}

\begin{Shaded}
\begin{Highlighting}[]
\CommentTok{\# Install new package}
\ExtensionTok{R}
\ExtensionTok{install.packages}\ErrorTok{(}\StringTok{"broom"}\KeywordTok{)}
\FunctionTok{quit()}

\CommentTok{\# Test the new functions}
\ExtensionTok{R}
\FunctionTok{devtools::load\_all()}
\FunctionTok{devtools::test()}
\FunctionTok{quit()}

\CommentTok{\# Create results directory and run enhanced analysis}
\FunctionTok{mkdir} \AttributeTok{{-}p}\NormalTok{ results}
\ExtensionTok{R} \AttributeTok{{-}{-}vanilla} \OperatorTok{\textless{}}\NormalTok{ scripts/01\_penguin\_exploration.R}

\CommentTok{\# Check what changed}
\FunctionTok{git}\NormalTok{ status}
\FunctionTok{git}\NormalTok{ diff}
\end{Highlighting}
\end{Shaded}

\subsubsection{Step 4: Second Commit with
Enhancement}\label{step-4-second-commit-with-enhancement}

\begin{Shaded}
\begin{Highlighting}[]
\CommentTok{\# Exit container}
\BuiltInTok{exit}

\CommentTok{\# Validate enhanced analysis}
\FunctionTok{make}\NormalTok{ docker{-}check{-}renv{-}fix    }\CommentTok{\# Track new broom package}
\FunctionTok{make}\NormalTok{ docker{-}test             }\CommentTok{\# Ensure all tests pass}

\CommentTok{\# Commit the enhancement}
\FunctionTok{git}\NormalTok{ add .}
\FunctionTok{git}\NormalTok{ commit }\AttributeTok{{-}m} \StringTok{"Add regression analysis to penguin bill study}

\StringTok{{-} Add linear regression line to scatter plot}
\StringTok{{-} Implement fit\_bill\_model() function with model diagnostics}
\StringTok{{-} Include R{-}squared and coefficient estimates}
\StringTok{{-} Add comprehensive tests for regression functionality  }
\StringTok{{-} Save model results for reproducibility}
\StringTok{{-} All tests passing, broom package added to dependencies"}

\FunctionTok{git}\NormalTok{ push origin main}
\end{Highlighting}
\end{Shaded}

\subsubsection{What This Example
Demonstrates}\label{what-this-example-demonstrates}

\begin{enumerate}
\def\labelenumi{\arabic{enumi}.}
\tightlist
\item
  \textbf{Complete workflow}: From initial analysis to enhanced version
\item
  \textbf{Professional practices}: Functions, tests, documentation
\item
  \textbf{Iterative development}: Build on previous work incrementally\\
\item
  \textbf{Dependency tracking}: Automatic renv.lock updates
\item
  \textbf{Reproducible outputs}: Saved plots and model objects
\item
  \textbf{Quality assurance}: Tests validate function behavior
\item
  \textbf{Version control}: Clear commit messages with detailed changes
\end{enumerate}

This example demonstrates the implementation of systematic data science
workflows using the ZZCOLLAB framework.

\subsection{Characteristics of Solo
Workflow}\label{characteristics-of-solo-workflow}

\begin{itemize}
\tightlist
\item
  \textbf{Reproducible}: Consistent computational environment across
  sessions
\item
  \textbf{Isolated}: Separation from host system R installation
\item
  \textbf{Extensible}: Compatible with collaborative workflows
\item
  \textbf{Validated}: Automated testing and continuous integration
\item
  \textbf{Configurable}: Multiple variants for different research
  applications
\item
  \textbf{Efficient}: Lightweight container options available
\item
  \textbf{Tracked}: Automated dependency management
\item
  \textbf{Versioned}: Complete project and environment history
\end{itemize}

\subsection{Solo to Team Transition}\label{solo-to-team-transition}

If you later want to collaborate:

\begin{enumerate}
\def\labelenumi{\arabic{enumi}.}
\item
  \textbf{Your project is already team-ready} - others can join with:

\begin{Shaded}
\begin{Highlighting}[]
\FunctionTok{git}\NormalTok{ clone https://github.com/yourname/myproject.git}
\BuiltInTok{cd}\NormalTok{ myproject}
\ExtensionTok{zzcollab} \AttributeTok{{-}t}\NormalTok{ yourname }\AttributeTok{{-}p}\NormalTok{ myproject }\AttributeTok{{-}I}\NormalTok{ analysis  }\CommentTok{\# Join with analysis variant}
\end{Highlighting}
\end{Shaded}
\item
  \textbf{No migration needed} - the same Docker images and workflow
  work for teams
\end{enumerate}

This workflow provides systematic reproducibility through standardized
development practices.

\subsection{Configuration System
Benefits}\label{configuration-system-benefits}

With configuration set up, all commands become simpler:

\begin{Shaded}
\begin{Highlighting}[]
\CommentTok{\# Traditional verbose approach:}
\ExtensionTok{zzcollab} \AttributeTok{{-}i} \AttributeTok{{-}t}\NormalTok{ rgt47 }\AttributeTok{{-}p}\NormalTok{ myproject }\AttributeTok{{-}B}\NormalTok{ analysis }\AttributeTok{{-}S} \AttributeTok{{-}d}\NormalTok{ \textasciitilde{}/dotfiles }\AttributeTok{{-}{-}github}

\CommentTok{\# Config{-}simplified approach (identical result):}
\ExtensionTok{zzcollab} \AttributeTok{{-}i} \AttributeTok{{-}p}\NormalTok{ myproject }\AttributeTok{{-}B}\NormalTok{ analysis }\AttributeTok{{-}{-}github}

\CommentTok{\# Modern variant approach (uses config.yaml):}
\ExtensionTok{zzcollab} \AttributeTok{{-}i} \AttributeTok{{-}p}\NormalTok{ myproject }\AttributeTok{{-}{-}github}    \CommentTok{\# Creates default variants automatically}
\end{Highlighting}
\end{Shaded}

The configuration system reduces parameter repetition while maintaining
workflow customization options.

\subsection{Using the R Interface}\label{using-the-r-interface}

\begin{Shaded}
\begin{Highlighting}[]
\CommentTok{\# Load the zzcollab R package}
\FunctionTok{library}\NormalTok{(zzcollab)}

\CommentTok{\# Configuration{-}based approach (recommended)}
\FunctionTok{init\_config}\NormalTok{()}
\FunctionTok{set\_config}\NormalTok{(}\StringTok{"team\_name"}\NormalTok{, }\StringTok{"myteam"}\NormalTok{)}
\FunctionTok{set\_config}\NormalTok{(}\StringTok{"build\_mode"}\NormalTok{, }\StringTok{"standard"}\NormalTok{)}

\CommentTok{\# Create and manage projects from R}
\FunctionTok{init\_project}\NormalTok{(}\AttributeTok{project\_name =} \StringTok{"penguin{-}analysis"}\NormalTok{)}
\FunctionTok{status}\NormalTok{()  }\CommentTok{\# Check project status}
\end{Highlighting}
\end{Shaded}

\subsection{Additional Resources}\label{additional-resources}

\begin{itemize}
\tightlist
\item
  \textbf{Team collaboration}: See
  \texttt{vignette("workflow-team",\ "zzcollab")} for multi-developer
  workflows
\item
  \textbf{Package development}: Explore the R package structure and
  testing framework
\item
  \textbf{Specialized variants}: Consider domain-specific variants for
  bioinformatics, geospatial analysis, or web applications
\item
  \textbf{Automation}: Utilize automated testing and Docker image
  management features
\end{itemize}

\end{document}
